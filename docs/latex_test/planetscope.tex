%% Generated by Sphinx.
\def\sphinxdocclass{report}
\documentclass[letterpaper,10pt,english]{sphinxmanual}
\ifdefined\pdfpxdimen
   \let\sphinxpxdimen\pdfpxdimen\else\newdimen\sphinxpxdimen
\fi \sphinxpxdimen=.75bp\relax
\ifdefined\pdfimageresolution
    \pdfimageresolution= \numexpr \dimexpr1in\relax/\sphinxpxdimen\relax
\fi
%% let collapsible pdf bookmarks panel have high depth per default
\PassOptionsToPackage{bookmarksdepth=5}{hyperref}

\PassOptionsToPackage{warn}{textcomp}
\usepackage[utf8]{inputenc}
\ifdefined\DeclareUnicodeCharacter
% support both utf8 and utf8x syntaxes
  \ifdefined\DeclareUnicodeCharacterAsOptional
    \def\sphinxDUC#1{\DeclareUnicodeCharacter{"#1}}
  \else
    \let\sphinxDUC\DeclareUnicodeCharacter
  \fi
  \sphinxDUC{00A0}{\nobreakspace}
  \sphinxDUC{2500}{\sphinxunichar{2500}}
  \sphinxDUC{2502}{\sphinxunichar{2502}}
  \sphinxDUC{2514}{\sphinxunichar{2514}}
  \sphinxDUC{251C}{\sphinxunichar{251C}}
  \sphinxDUC{2572}{\textbackslash}
\fi
\usepackage{cmap}
\usepackage[T1]{fontenc}
\usepackage{amsmath,amssymb,amstext}
\usepackage{babel}



\usepackage{tgtermes}
\usepackage{tgheros}
\renewcommand{\ttdefault}{txtt}



\usepackage[Bjarne]{fncychap}
\usepackage{sphinx}

\fvset{fontsize=auto}
\usepackage{geometry}


% Include hyperref last.
\usepackage{hyperref}
% Fix anchor placement for figures with captions.
\usepackage{hypcap}% it must be loaded after hyperref.
% Set up styles of URL: it should be placed after hyperref.
\urlstyle{same}


\usepackage{sphinxmessages}



% Jupyter Notebook code cell colors
\definecolor{nbsphinxin}{HTML}{307FC1}
\definecolor{nbsphinxout}{HTML}{BF5B3D}
\definecolor{nbsphinx-code-bg}{HTML}{F5F5F5}
\definecolor{nbsphinx-code-border}{HTML}{E0E0E0}
\definecolor{nbsphinx-stderr}{HTML}{FFDDDD}
% ANSI colors for output streams and traceback highlighting
\definecolor{ansi-black}{HTML}{3E424D}
\definecolor{ansi-black-intense}{HTML}{282C36}
\definecolor{ansi-red}{HTML}{E75C58}
\definecolor{ansi-red-intense}{HTML}{B22B31}
\definecolor{ansi-green}{HTML}{00A250}
\definecolor{ansi-green-intense}{HTML}{007427}
\definecolor{ansi-yellow}{HTML}{DDB62B}
\definecolor{ansi-yellow-intense}{HTML}{B27D12}
\definecolor{ansi-blue}{HTML}{208FFB}
\definecolor{ansi-blue-intense}{HTML}{0065CA}
\definecolor{ansi-magenta}{HTML}{D160C4}
\definecolor{ansi-magenta-intense}{HTML}{A03196}
\definecolor{ansi-cyan}{HTML}{60C6C8}
\definecolor{ansi-cyan-intense}{HTML}{258F8F}
\definecolor{ansi-white}{HTML}{C5C1B4}
\definecolor{ansi-white-intense}{HTML}{A1A6B2}
\definecolor{ansi-default-inverse-fg}{HTML}{FFFFFF}
\definecolor{ansi-default-inverse-bg}{HTML}{000000}

% Define an environment for non-plain-text code cell outputs (e.g. images)
\makeatletter
\newenvironment{nbsphinxfancyoutput}{%
    % Avoid fatal error with framed.sty if graphics too long to fit on one page
    \let\sphinxincludegraphics\nbsphinxincludegraphics
    \nbsphinx@image@maxheight\textheight
    \advance\nbsphinx@image@maxheight -2\fboxsep   % default \fboxsep 3pt
    \advance\nbsphinx@image@maxheight -2\fboxrule  % default \fboxrule 0.4pt
    \advance\nbsphinx@image@maxheight -\baselineskip
\def\nbsphinxfcolorbox{\spx@fcolorbox{nbsphinx-code-border}{white}}%
\def\FrameCommand{\nbsphinxfcolorbox\nbsphinxfancyaddprompt\@empty}%
\def\FirstFrameCommand{\nbsphinxfcolorbox\nbsphinxfancyaddprompt\sphinxVerbatim@Continues}%
\def\MidFrameCommand{\nbsphinxfcolorbox\sphinxVerbatim@Continued\sphinxVerbatim@Continues}%
\def\LastFrameCommand{\nbsphinxfcolorbox\sphinxVerbatim@Continued\@empty}%
\MakeFramed{\advance\hsize-\width\@totalleftmargin\z@\linewidth\hsize\@setminipage}%
\lineskip=1ex\lineskiplimit=1ex\raggedright%
}{\par\unskip\@minipagefalse\endMakeFramed}
\makeatother
\newbox\nbsphinxpromptbox
\def\nbsphinxfancyaddprompt{\ifvoid\nbsphinxpromptbox\else
    \kern\fboxrule\kern\fboxsep
    \copy\nbsphinxpromptbox
    \kern-\ht\nbsphinxpromptbox\kern-\dp\nbsphinxpromptbox
    \kern-\fboxsep\kern-\fboxrule\nointerlineskip
    \fi}
\newlength\nbsphinxcodecellspacing
\setlength{\nbsphinxcodecellspacing}{0pt}

% Define support macros for attaching opening and closing lines to notebooks
\newsavebox\nbsphinxbox
\makeatletter
\newcommand{\nbsphinxstartnotebook}[1]{%
    \par
    % measure needed space
    \setbox\nbsphinxbox\vtop{{#1\par}}
    % reserve some space at bottom of page, else start new page
    \needspace{\dimexpr2.5\baselineskip+\ht\nbsphinxbox+\dp\nbsphinxbox}
    % mimick vertical spacing from \section command
      \addpenalty\@secpenalty
      \@tempskipa 3.5ex \@plus 1ex \@minus .2ex\relax
      \addvspace\@tempskipa
      {\Large\@tempskipa\baselineskip
             \advance\@tempskipa-\prevdepth
             \advance\@tempskipa-\ht\nbsphinxbox
             \ifdim\@tempskipa>\z@
               \vskip \@tempskipa
             \fi}
    \unvbox\nbsphinxbox
    % if notebook starts with a \section, prevent it from adding extra space
    \@nobreaktrue\everypar{\@nobreakfalse\everypar{}}%
    % compensate the parskip which will get inserted by next paragraph
    \nobreak\vskip-\parskip
    % do not break here
    \nobreak
}% end of \nbsphinxstartnotebook

\newcommand{\nbsphinxstopnotebook}[1]{%
    \par
    % measure needed space
    \setbox\nbsphinxbox\vbox{{#1\par}}
    \nobreak % it updates page totals
    \dimen@\pagegoal
    \advance\dimen@-\pagetotal \advance\dimen@-\pagedepth
    \advance\dimen@-\ht\nbsphinxbox \advance\dimen@-\dp\nbsphinxbox
    \ifdim\dimen@<\z@
      % little space left
      \unvbox\nbsphinxbox
      \kern-.8\baselineskip
      \nobreak\vskip\z@\@plus1fil
      \penalty100
      \vskip\z@\@plus-1fil
      \kern.8\baselineskip
    \else
      \unvbox\nbsphinxbox
    \fi
}% end of \nbsphinxstopnotebook

% Ensure height of an included graphics fits in nbsphinxfancyoutput frame
\newdimen\nbsphinx@image@maxheight % set in nbsphinxfancyoutput environment
\newcommand*{\nbsphinxincludegraphics}[2][]{%
    \gdef\spx@includegraphics@options{#1}%
    \setbox\spx@image@box\hbox{\includegraphics[#1,draft]{#2}}%
    \in@false
    \ifdim \wd\spx@image@box>\linewidth
      \g@addto@macro\spx@includegraphics@options{,width=\linewidth}%
      \in@true
    \fi
    % no rotation, no need to worry about depth
    \ifdim \ht\spx@image@box>\nbsphinx@image@maxheight
      \g@addto@macro\spx@includegraphics@options{,height=\nbsphinx@image@maxheight}%
      \in@true
    \fi
    \ifin@
      \g@addto@macro\spx@includegraphics@options{,keepaspectratio}%
    \fi
    \setbox\spx@image@box\box\voidb@x % clear memory
    \expandafter\includegraphics\expandafter[\spx@includegraphics@options]{#2}%
}% end of "\MakeFrame"-safe variant of \sphinxincludegraphics
\makeatother

\makeatletter
\renewcommand*\sphinx@verbatim@nolig@list{\do\'\do\`}
\begingroup
\catcode`'=\active
\let\nbsphinx@noligs\@noligs
\g@addto@macro\nbsphinx@noligs{\let'\PYGZsq}
\endgroup
\makeatother
\renewcommand*\sphinxbreaksbeforeactivelist{\do\<\do\"\do\'}
\renewcommand*\sphinxbreaksafteractivelist{\do\.\do\,\do\:\do\;\do\?\do\!\do\/\do\>\do\-}
\makeatletter
\fvset{codes*=\sphinxbreaksattexescapedchars\do\^\^\let\@noligs\nbsphinx@noligs}
\makeatother



\title{PlanetScope}
\date{Dec 31, 2022}
\release{0.1}
\author{CSDAP Developers}
\newcommand{\sphinxlogo}{\vbox{}}
\renewcommand{\releasename}{Release}
\makeindex
\begin{document}

\pagestyle{empty}
\sphinxmaketitle
\pagestyle{plain}
\sphinxtableofcontents
\pagestyle{normal}
\phantomsection\label{\detokenize{index::doc}}


\sphinxAtStartPar
The planetscope package is a tool for reading, plotting, and processing data from the PlanetScope satellite imaging system. It is available on PyPI and can be installed using pip.


\chapter{Installation}
\label{\detokenize{gettingstarted:installation}}
\sphinxAtStartPar
To install the latest version of planetscope from PyPI, use the following command:

\begin{sphinxVerbatim}[commandchars=\\\{\}]
pip install planetscope
\end{sphinxVerbatim}

\sphinxAtStartPar
Alternatively, you can install the package from the source code by cloning the Git repository and using pip to install it in editable mode:

\begin{sphinxVerbatim}[commandchars=\\\{\}]
git clone https://github.com/NASA\PYGZhy{}IMPACT/planet\PYGZus{}utils
\PYG{n+nb}{cd} planetscope
python3 setup.py install
\PYG{n+nb}{cd} ..
\end{sphinxVerbatim}

\sphinxAtStartPar
The PlanetScope package can be installed with pip directly from the github:

\begin{sphinxVerbatim}[commandchars=\\\{\}]
pip install git+https://github.com/NASA\PYGZhy{}IMPACT/planet\PYGZus{}utils
\end{sphinxVerbatim}


\chapter{Dependencies}
\label{\detokenize{gettingstarted:dependencies}}
\sphinxAtStartPar
The planetscope package has the following dependencies:
\begin{itemize}
\item {} 
\sphinxAtStartPar
NumPy

\item {} 
\sphinxAtStartPar
GDAL

\item {} 
\sphinxAtStartPar
Matplotlib

\item {} 
\sphinxAtStartPar
Scikit\sphinxhyphen{}learn (optional, for generating training data)

\end{itemize}

\sphinxAtStartPar
These dependencies will be installed automatically when you install the planetscope package using pip.


\chapter{Usage}
\label{\detokenize{gettingstarted:usage}}
\sphinxAtStartPar
Here is a simple example of how to use the planetscope package to read and plot a PlanetScope image:

\begin{sphinxVerbatim}[commandchars=\\\{\}]
\PYG{k+kn}{import} \PYG{n+nn}{planetscope} \PYG{k}{as} \PYG{n+nn}{ps}
\PYG{c+c1}{\PYGZsh{}Read in the image data}
\PYG{n}{data}\PYG{p}{,} \PYG{n}{metadata} \PYG{o}{=} \PYG{n}{ps}\PYG{o}{.}\PYG{n}{read}\PYG{p}{(}\PYG{l+s+s2}{\PYGZdq{}}\PYG{l+s+s2}{path/to/image.tif}\PYG{l+s+s2}{\PYGZdq{}}\PYG{p}{)}
\PYG{c+c1}{\PYGZsh{}Plot the image}
\PYG{n}{ps}\PYG{o}{.}\PYG{n}{plot}\PYG{p}{(}\PYG{n}{data}\PYG{p}{)}
\end{sphinxVerbatim}

\sphinxAtStartPar
For more information on the available functions and how to use them, see the documentation for each module in the package:
\begin{quote}
\begin{quote}\begin{description}
\item[{mod:planetscope.reader}] \leavevmode
\sphinxAtStartPar
functions for reading and parsing the raw data files.

\item[{mod:planetscope.plotter}] \leavevmode
\sphinxAtStartPar
functions for generating plots and maps from the data.

\item[{mod:planetscope.processor}] \leavevmode
\sphinxAtStartPar
functions for processing the data, such as image cropping, resampling, and band math.

\item[{mod:planetscope.training\_data}] \leavevmode
\sphinxAtStartPar
functions for generating training data for machine learning algorithms.

\end{description}\end{quote}
\end{quote}


\chapter{Support}
\label{\detokenize{gettingstarted:support}}
\sphinxAtStartPar
If you have any questions or encounter any issues while using the planetscope package, please file an issue on the package’s GitHub repository: \sphinxurl{https://github.com/NASA-IMPACT/planet\_utils/issues}.


\chapter{Tutorials}
\label{\detokenize{tutorials:tutorials}}\label{\detokenize{tutorials::doc}}
\sphinxAtStartPar
For more information on Jupyter itself, see \sphinxhref{https://jupyter.org/}{their web site}.


\section{Introducing Jupyter Notebooks}
\label{\detokenize{notebooks/intro:Introducing-Jupyter-Notebooks}}\label{\detokenize{notebooks/intro::doc}}
\sphinxAtStartPar
First, set up the environment:

\begin{sphinxuseclass}{nbinput}
\begin{sphinxuseclass}{nblast}
{
\sphinxsetup{VerbatimColor={named}{nbsphinx-code-bg}}
\sphinxsetup{VerbatimBorderColor={named}{nbsphinx-code-border}}
\begin{sphinxVerbatim}[commandchars=\\\{\}]
\llap{\color{nbsphinxin}[1]:\,\hspace{\fboxrule}\hspace{\fboxsep}}\PYG{k+kn}{import} \PYG{n+nn}{matplotlib}
\PYG{k+kn}{import} \PYG{n+nn}{matplotlib}\PYG{n+nn}{.}\PYG{n+nn}{pyplot} \PYG{k}{as} \PYG{n+nn}{pl}
\PYG{k+kn}{import} \PYG{n+nn}{numpy} \PYG{k}{as} \PYG{n+nn}{np}

\PYG{k}{try}\PYG{p}{:}
    \PYG{k+kn}{from} \PYG{n+nn}{IPython} \PYG{k+kn}{import} \PYG{n}{get\PYGZus{}ipython}
    \PYG{n}{get\PYGZus{}ipython}\PYG{p}{(}\PYG{p}{)}\PYG{o}{.}\PYG{n}{run\PYGZus{}line\PYGZus{}magic}\PYG{p}{(}\PYG{l+s+s1}{\PYGZsq{}}\PYG{l+s+s1}{matplotlib}\PYG{l+s+s1}{\PYGZsq{}}\PYG{p}{,} \PYG{l+s+s1}{\PYGZsq{}}\PYG{l+s+s1}{inline}\PYG{l+s+s1}{\PYGZsq{}}\PYG{p}{)}
\PYG{k}{except} \PYG{n+ne}{AttributeError}\PYG{p}{:}
    \PYG{n+nb}{print}\PYG{p}{(}\PYG{l+s+s1}{\PYGZsq{}}\PYG{l+s+s1}{Magic function can only be used in IPython environment}\PYG{l+s+s1}{\PYGZsq{}}\PYG{p}{)}
    \PYG{n}{matplotlib}\PYG{o}{.}\PYG{n}{use}\PYG{p}{(}\PYG{l+s+s1}{\PYGZsq{}}\PYG{l+s+s1}{Agg}\PYG{l+s+s1}{\PYGZsq{}}\PYG{p}{)}

\PYG{n}{pl}\PYG{o}{.}\PYG{n}{rcParams}\PYG{p}{[}\PYG{l+s+s2}{\PYGZdq{}}\PYG{l+s+s2}{figure.figsize}\PYG{l+s+s2}{\PYGZdq{}}\PYG{p}{]} \PYG{o}{=} \PYG{p}{[}\PYG{l+m+mi}{15}\PYG{p}{,} \PYG{l+m+mi}{8}\PYG{p}{]}
\end{sphinxVerbatim}
}

\end{sphinxuseclass}
\end{sphinxuseclass}
\sphinxAtStartPar
Then, define a function that creates a pretty graph:

\begin{sphinxuseclass}{nbinput}
\begin{sphinxuseclass}{nblast}
{
\sphinxsetup{VerbatimColor={named}{nbsphinx-code-bg}}
\sphinxsetup{VerbatimBorderColor={named}{nbsphinx-code-border}}
\begin{sphinxVerbatim}[commandchars=\\\{\}]
\llap{\color{nbsphinxin}[2]:\,\hspace{\fboxrule}\hspace{\fboxsep}}\PYG{k}{def} \PYG{n+nf}{SineAndCosineWaves}\PYG{p}{(}\PYG{p}{)}\PYG{p}{:}
    \PYG{c+c1}{\PYGZsh{} Get a large number of X values for a nice smooth curve. Using Pi as np.sin requires radians...}
    \PYG{n}{x} \PYG{o}{=} \PYG{n}{np}\PYG{o}{.}\PYG{n}{linspace}\PYG{p}{(}\PYG{l+m+mi}{0}\PYG{p}{,} \PYG{l+m+mi}{2} \PYG{o}{*} \PYG{n}{np}\PYG{o}{.}\PYG{n}{pi}\PYG{p}{,} \PYG{l+m+mi}{180}\PYG{p}{)}
    \PYG{c+c1}{\PYGZsh{} Convert radians to degrees to make for a meaningful X axis (1 radian = 57.29* degrees)}
    \PYG{n}{xdeg} \PYG{o}{=} \PYG{l+m+mf}{57.29577951308232} \PYG{o}{*} \PYG{n}{np}\PYG{o}{.}\PYG{n}{array}\PYG{p}{(}\PYG{n}{x}\PYG{p}{)}
    \PYG{c+c1}{\PYGZsh{} Calculate the sine of each value of X}
    \PYG{n}{y} \PYG{o}{=} \PYG{n}{np}\PYG{o}{.}\PYG{n}{sin}\PYG{p}{(}\PYG{n}{x}\PYG{p}{)}
    \PYG{c+c1}{\PYGZsh{} Calculate the cosine of each value of X}
    \PYG{n}{z} \PYG{o}{=} \PYG{n}{np}\PYG{o}{.}\PYG{n}{cos}\PYG{p}{(}\PYG{n}{x}\PYG{p}{)}
    \PYG{c+c1}{\PYGZsh{} Plot the sine wave in blue, using degrees rather than radians on the X axis}
    \PYG{n}{pl}\PYG{o}{.}\PYG{n}{plot}\PYG{p}{(}\PYG{n}{xdeg}\PYG{p}{,} \PYG{n}{y}\PYG{p}{,} \PYG{n}{color}\PYG{o}{=}\PYG{l+s+s1}{\PYGZsq{}}\PYG{l+s+s1}{blue}\PYG{l+s+s1}{\PYGZsq{}}\PYG{p}{,} \PYG{n}{label}\PYG{o}{=}\PYG{l+s+s1}{\PYGZsq{}}\PYG{l+s+s1}{Sine wave}\PYG{l+s+s1}{\PYGZsq{}}\PYG{p}{)}
    \PYG{c+c1}{\PYGZsh{} Plot the cos wave in green, using degrees rather than radians on the X axis}
    \PYG{n}{pl}\PYG{o}{.}\PYG{n}{plot}\PYG{p}{(}\PYG{n}{xdeg}\PYG{p}{,} \PYG{n}{z}\PYG{p}{,} \PYG{n}{color}\PYG{o}{=}\PYG{l+s+s1}{\PYGZsq{}}\PYG{l+s+s1}{green}\PYG{l+s+s1}{\PYGZsq{}}\PYG{p}{,} \PYG{n}{label}\PYG{o}{=}\PYG{l+s+s1}{\PYGZsq{}}\PYG{l+s+s1}{Cosine wave}\PYG{l+s+s1}{\PYGZsq{}}\PYG{p}{)}
    \PYG{n}{pl}\PYG{o}{.}\PYG{n}{xlabel}\PYG{p}{(}\PYG{l+s+s2}{\PYGZdq{}}\PYG{l+s+s2}{Degrees}\PYG{l+s+s2}{\PYGZdq{}}\PYG{p}{)}
    \PYG{c+c1}{\PYGZsh{} More sensible X axis values}
    \PYG{n}{pl}\PYG{o}{.}\PYG{n}{xticks}\PYG{p}{(}\PYG{n}{np}\PYG{o}{.}\PYG{n}{arange}\PYG{p}{(}\PYG{l+m+mi}{0}\PYG{p}{,} \PYG{l+m+mi}{361}\PYG{p}{,} \PYG{l+m+mi}{45}\PYG{p}{)}\PYG{p}{)}
    \PYG{n}{pl}\PYG{o}{.}\PYG{n}{legend}\PYG{p}{(}\PYG{p}{)}
    \PYG{n}{pl}\PYG{o}{.}\PYG{n}{show}\PYG{p}{(}\PYG{p}{)}
\end{sphinxVerbatim}
}

\end{sphinxuseclass}
\end{sphinxuseclass}
\sphinxAtStartPar
Finally, call that function to display the graph:

\begin{sphinxuseclass}{nbinput}
{
\sphinxsetup{VerbatimColor={named}{nbsphinx-code-bg}}
\sphinxsetup{VerbatimBorderColor={named}{nbsphinx-code-border}}
\begin{sphinxVerbatim}[commandchars=\\\{\}]
\llap{\color{nbsphinxin}[3]:\,\hspace{\fboxrule}\hspace{\fboxsep}}\PYG{n}{SineAndCosineWaves}\PYG{p}{(}\PYG{p}{)}
\end{sphinxVerbatim}
}

\end{sphinxuseclass}
\begin{sphinxuseclass}{nboutput}
\begin{sphinxuseclass}{nblast}
\hrule height -\fboxrule\relax
\vspace{\nbsphinxcodecellspacing}

\makeatletter\setbox\nbsphinxpromptbox\box\voidb@x\makeatother

\begin{nbsphinxfancyoutput}

\begin{sphinxuseclass}{output_area}
\begin{sphinxuseclass}{}
\noindent\sphinxincludegraphics[width=889\sphinxpxdimen,height=480\sphinxpxdimen]{{notebooks_intro_5_0}.png}

\end{sphinxuseclass}
\end{sphinxuseclass}
\end{nbsphinxfancyoutput}

\end{sphinxuseclass}
\end{sphinxuseclass}

\chapter{planetscope}
\label{\detokenize{_autosummary/planetscope:module-planetscope}}\label{\detokenize{_autosummary/planetscope:planetscope}}\label{\detokenize{_autosummary/planetscope::doc}}\index{module@\spxentry{module}!planetscope@\spxentry{planetscope}}\index{planetscope@\spxentry{planetscope}!module@\spxentry{module}}
\sphinxAtStartPar
API reference documentation for the example \sphinxtitleref{mytoolbox} package.


\begin{savenotes}\sphinxatlongtablestart\begin{longtable}[c]{\X{1}{2}\X{1}{2}}
\hline

\endfirsthead

\multicolumn{2}{c}%
{\makebox[0pt]{\sphinxtablecontinued{\tablename\ \thetable{} \textendash{} continued from previous page}}}\\
\hline

\endhead

\hline
\multicolumn{2}{r}{\makebox[0pt][r]{\sphinxtablecontinued{continues on next page}}}\\
\endfoot

\endlastfoot

\sphinxAtStartPar
{\hyperref[\detokenize{_autosummary/planetscope.ahps:module-planetscope.ahps}]{\sphinxcrossref{\sphinxcode{\sphinxupquote{planetscope.ahps}}}}}
&
\sphinxAtStartPar

\\
\hline
\sphinxAtStartPar
{\hyperref[\detokenize{_autosummary/planetscope.coast:module-planetscope.coast}]{\sphinxcrossref{\sphinxcode{\sphinxupquote{planetscope.coast}}}}}
&
\sphinxAtStartPar

\\
\hline
\sphinxAtStartPar
{\hyperref[\detokenize{_autosummary/planetscope.mymodule1:module-planetscope.mymodule1}]{\sphinxcrossref{\sphinxcode{\sphinxupquote{planetscope.mymodule1}}}}}
&
\sphinxAtStartPar
Module containing a class, global function, exception, type variable and type alias.
\\
\hline
\sphinxAtStartPar
{\hyperref[\detokenize{_autosummary/planetscope.mymodule2:module-planetscope.mymodule2}]{\sphinxcrossref{\sphinxcode{\sphinxupquote{planetscope.mymodule2}}}}}
&
\sphinxAtStartPar
Module containing a class derived from another class.
\\
\hline
\sphinxAtStartPar
{\hyperref[\detokenize{_autosummary/planetscope.mymodule3a:module-planetscope.mymodule3a}]{\sphinxcrossref{\sphinxcode{\sphinxupquote{planetscope.mymodule3a}}}}}
&
\sphinxAtStartPar
Module containing a class derived from another class.
\\
\hline
\sphinxAtStartPar
{\hyperref[\detokenize{_autosummary/planetscope.mysubpackage:module-planetscope.mysubpackage}]{\sphinxcrossref{\sphinxcode{\sphinxupquote{planetscope.mysubpackage}}}}}
&
\sphinxAtStartPar
Package demonstrating one level of nesting.
\\
\hline
\end{longtable}\sphinxatlongtableend\end{savenotes}


\section{planetscope.ahps}
\label{\detokenize{_autosummary/planetscope.ahps:module-planetscope.ahps}}\label{\detokenize{_autosummary/planetscope.ahps:planetscope-ahps}}\label{\detokenize{_autosummary/planetscope.ahps::doc}}\index{module@\spxentry{module}!planetscope.ahps@\spxentry{planetscope.ahps}}\index{planetscope.ahps@\spxentry{planetscope.ahps}!module@\spxentry{module}}\subsubsection*{Functions}


\begin{savenotes}\sphinxatlongtablestart\begin{longtable}[c]{\X{1}{2}\X{1}{2}}
\hline

\endfirsthead

\multicolumn{2}{c}%
{\makebox[0pt]{\sphinxtablecontinued{\tablename\ \thetable{} \textendash{} continued from previous page}}}\\
\hline

\endhead

\hline
\multicolumn{2}{r}{\makebox[0pt][r]{\sphinxtablecontinued{continues on next page}}}\\
\endfoot

\endlastfoot

\sphinxAtStartPar
{\hyperref[\detokenize{_autosummary/planetscope.ahps.read_ahps:planetscope.ahps.read_ahps}]{\sphinxcrossref{\sphinxcode{\sphinxupquote{read\_ahps}}}}}
&
\sphinxAtStartPar
\_summary\_
\\
\hline
\end{longtable}\sphinxatlongtableend\end{savenotes}


\subsection{planetscope.ahps.read\_ahps}
\label{\detokenize{_autosummary/planetscope.ahps.read_ahps:planetscope-ahps-read-ahps}}\label{\detokenize{_autosummary/planetscope.ahps.read_ahps::doc}}\index{read\_ahps() (in module planetscope.ahps)@\spxentry{read\_ahps()}\spxextra{in module planetscope.ahps}}

\begin{fulllineitems}
\phantomsection\label{\detokenize{_autosummary/planetscope.ahps.read_ahps:planetscope.ahps.read_ahps}}\pysiglinewithargsret{\sphinxbfcode{\sphinxupquote{read\_ahps}}}{\emph{\DUrole{n}{ahps\_filename}}}{}
\sphinxAtStartPar
\_summary\_

\sphinxAtStartPar
Read and process the AHPS (Advanced Hydrologic Prediction Service) dataset from a given file. The AHPS dataset is
transformed to Plate Carree projection and invalid observation values are replaced with NaN.
\begin{description}
\item[{Args:}] \leavevmode\begin{description}
\item[{ahps\_filename}] \leavevmode{[}str{]}
\sphinxAtStartPar
Filename for the AHPS dataset.

\end{description}

\item[{Returns:}] \leavevmode
\sphinxAtStartPar
\_type\_: \_description\_
\begin{description}
\item[{lon\_platecarre}] \leavevmode{[}numpy.ndarray{]}
\sphinxAtStartPar
Longitude values in Plate Carree projection.

\item[{lat\_platecarre}] \leavevmode{[}numpy.ndarray{]}
\sphinxAtStartPar
Latitude values in Plate Carree projection.

\item[{obs}] \leavevmode{[}numpy.ndarray{]}
\sphinxAtStartPar
Observation values from the AHPS dataset, with invalid values replaced with NaN and converted to millimeters.

\end{description}

\sphinxAtStartPar
Example:

\begin{sphinxVerbatim}[commandchars=\\\{\}]
\PYG{g+gp}{\PYGZgt{}\PYGZgt{}\PYGZgt{} }\PYG{n}{read\PYGZus{}ahps}\PYG{p}{(}\PYG{l+s+s2}{\PYGZdq{}}\PYG{l+s+s2}{my\PYGZus{}ahps\PYGZus{}dataset.nc}\PYG{l+s+s2}{\PYGZdq{}}\PYG{p}{)}
\PYG{g+go}{(array([[\PYGZhy{}180., \PYGZhy{}178., ...,  178.,  180.],}
\PYG{g+go}{        [\PYGZhy{}180., \PYGZhy{}178., ...,  178.,  180.],}
\PYG{g+go}{        ...,}
\PYG{g+go}{        [\PYGZhy{}180., \PYGZhy{}178., ...,  178.,  180.],}
\PYG{g+go}{        [\PYGZhy{}180., \PYGZhy{}178., ...,  178.,  180.]]),}
\PYG{g+go}{ array([[\PYGZhy{}90., \PYGZhy{}90., ..., \PYGZhy{}90., \PYGZhy{}90.],}
\PYG{g+go}{        [\PYGZhy{}88., \PYGZhy{}88., ..., \PYGZhy{}88., \PYGZhy{}88.],}
\PYG{g+go}{        ...,}
\PYG{g+go}{        [ 88.,  88., ...,  88.,  88.],}
\PYG{g+go}{        [ 90.,  90., ...,  90.,  90.]]),}
\PYG{g+go}{ array([[nan, nan, ..., nan, nan],}
\PYG{g+go}{        [nan, nan, ..., nan, nan],}
\PYG{g+go}{        ...,}
\PYG{g+go}{        [nan, nan, ..., nan, nan],}
\PYG{g+go}{        [nan, nan, ..., nan, nan]]))}
\end{sphinxVerbatim}

\end{description}

\end{fulllineitems}



\section{planetscope.coast}
\label{\detokenize{_autosummary/planetscope.coast:module-planetscope.coast}}\label{\detokenize{_autosummary/planetscope.coast:planetscope-coast}}\label{\detokenize{_autosummary/planetscope.coast::doc}}\index{module@\spxentry{module}!planetscope.coast@\spxentry{planetscope.coast}}\index{planetscope.coast@\spxentry{planetscope.coast}!module@\spxentry{module}}\subsubsection*{Functions}


\begin{savenotes}\sphinxatlongtablestart\begin{longtable}[c]{\X{1}{2}\X{1}{2}}
\hline

\endfirsthead

\multicolumn{2}{c}%
{\makebox[0pt]{\sphinxtablecontinued{\tablename\ \thetable{} \textendash{} continued from previous page}}}\\
\hline

\endhead

\hline
\multicolumn{2}{r}{\makebox[0pt][r]{\sphinxtablecontinued{continues on next page}}}\\
\endfoot

\endlastfoot

\sphinxAtStartPar
{\hyperref[\detokenize{_autosummary/planetscope.coast.plot_coast:planetscope.coast.plot_coast}]{\sphinxcrossref{\sphinxcode{\sphinxupquote{plot\_coast}}}}}
&
\sphinxAtStartPar
Plot natural features and gridlines on a map using Cartopy.
\\
\hline
\end{longtable}\sphinxatlongtableend\end{savenotes}


\subsection{planetscope.coast.plot\_coast}
\label{\detokenize{_autosummary/planetscope.coast.plot_coast:planetscope-coast-plot-coast}}\label{\detokenize{_autosummary/planetscope.coast.plot_coast::doc}}\index{plot\_coast() (in module planetscope.coast)@\spxentry{plot\_coast()}\spxextra{in module planetscope.coast}}

\begin{fulllineitems}
\phantomsection\label{\detokenize{_autosummary/planetscope.coast.plot_coast:planetscope.coast.plot_coast}}\pysiglinewithargsret{\sphinxbfcode{\sphinxupquote{plot\_coast}}}{\emph{\DUrole{n}{axes}}}{}
\sphinxAtStartPar
Plot natural features and gridlines on a map using Cartopy.
\begin{description}
\item[{Args:}] \leavevmode
\sphinxAtStartPar
axes (cartopy.mpl.geoaxes.GeoAxes): The axes object to plot on.

\item[{Returns:}] \leavevmode
\sphinxAtStartPar
\_type\_: \_description\_

\end{description}

\sphinxAtStartPar
Example:
\textgreater{}\textgreater{}\textgreater{} import matplotlib.pyplot as plt
\textgreater{}\textgreater{}\textgreater{} import cartopy.crs as ccrs

\begin{sphinxVerbatim}[commandchars=\\\{\}]
\PYG{g+gp}{\PYGZgt{}\PYGZgt{}\PYGZgt{} }\PYG{c+c1}{\PYGZsh{}Set up the figure and axes}
\PYG{g+gp}{\PYGZgt{}\PYGZgt{}\PYGZgt{} }\PYG{n}{fig}\PYG{p}{,} \PYG{n}{ax} \PYG{o}{=} \PYG{n}{plt}\PYG{o}{.}\PYG{n}{subplots}\PYG{p}{(}\PYG{l+m+mi}{1}\PYG{p}{,} \PYG{l+m+mi}{1}\PYG{p}{,} \PYG{n}{figsize}\PYG{o}{=}\PYG{p}{(}\PYG{l+m+mi}{10}\PYG{p}{,} \PYG{l+m+mi}{10}\PYG{p}{)}\PYG{p}{)}
\end{sphinxVerbatim}

\begin{sphinxVerbatim}[commandchars=\\\{\}]
\PYG{g+gp}{\PYGZgt{}\PYGZgt{}\PYGZgt{} }\PYG{c+c1}{\PYGZsh{}Set the projection for the axes}
\PYG{g+gp}{\PYGZgt{}\PYGZgt{}\PYGZgt{} }\PYG{n}{ax}\PYG{o}{.}\PYG{n}{projection} \PYG{o}{=} \PYG{n}{ccrs}\PYG{o}{.}\PYG{n}{PlateCarree}\PYG{p}{(}\PYG{p}{)}
\end{sphinxVerbatim}

\begin{sphinxVerbatim}[commandchars=\\\{\}]
\PYG{g+gp}{\PYGZgt{}\PYGZgt{}\PYGZgt{} }\PYG{c+c1}{\PYGZsh{}Call the plot\PYGZus{}coast function}
\PYG{g+gp}{\PYGZgt{}\PYGZgt{}\PYGZgt{} }\PYG{n}{plot\PYGZus{}coast}\PYG{p}{(}\PYG{n}{ax}\PYG{p}{)}
\end{sphinxVerbatim}

\begin{sphinxVerbatim}[commandchars=\\\{\}]
\PYG{g+gp}{\PYGZgt{}\PYGZgt{}\PYGZgt{} }\PYG{c+c1}{\PYGZsh{}Show the plot}
\PYG{g+gp}{\PYGZgt{}\PYGZgt{}\PYGZgt{} }\PYG{n}{plt}\PYG{o}{.}\PYG{n}{show}\PYG{p}{(}\PYG{p}{)}
\end{sphinxVerbatim}
\begin{quote}\begin{description}
\item[{Return type}] \leavevmode
\sphinxAtStartPar
\sphinxhref{https://docs.python.org/3/library/constants.html\#None}{\sphinxcode{\sphinxupquote{None}}}

\end{description}\end{quote}

\end{fulllineitems}



\section{planetscope.mymodule1}
\label{\detokenize{_autosummary/planetscope.mymodule1:module-planetscope.mymodule1}}\label{\detokenize{_autosummary/planetscope.mymodule1:planetscope-mymodule1}}\label{\detokenize{_autosummary/planetscope.mymodule1::doc}}\index{module@\spxentry{module}!planetscope.mymodule1@\spxentry{planetscope.mymodule1}}\index{planetscope.mymodule1@\spxentry{planetscope.mymodule1}!module@\spxentry{module}}
\sphinxAtStartPar
Module containing a class, global function, exception, type variable and type alias.

\sphinxAtStartPar
The type variable gets included in the list of variables captured by the template and so is
automatically summarised and documented.

\sphinxAtStartPar
For some reason, the type alias doesn’t. However, it can be documented manually using
the \sphinxtitleref{autodata} directive, ie:

\begin{sphinxVerbatim}[commandchars=\\\{\}]
\PYG{o}{.}\PYG{o}{.} \PYG{n}{autodata}\PYG{p}{:}\PYG{p}{:} \PYG{n}{myTypeAlias}
\end{sphinxVerbatim}
\index{myTypeAlias (in module planetscope.mymodule1)@\spxentry{myTypeAlias}\spxextra{in module planetscope.mymodule1}}

\begin{fulllineitems}
\phantomsection\label{\detokenize{_autosummary/planetscope.mymodule1:planetscope.mymodule1.myTypeAlias}}\pysigline{\sphinxbfcode{\sphinxupquote{myTypeAlias}}\sphinxbfcode{\sphinxupquote{\DUrole{w}{  }\DUrole{p}{=}\DUrole{w}{  }typing.Union{[}typing.List{[}str{]}, typing.Tuple{[}int, int{]}{]}}}}
\sphinxAtStartPar
This is a type alias.

\end{fulllineitems}

\subsubsection*{Functions}


\begin{savenotes}\sphinxatlongtablestart\begin{longtable}[c]{\X{1}{2}\X{1}{2}}
\hline

\endfirsthead

\multicolumn{2}{c}%
{\makebox[0pt]{\sphinxtablecontinued{\tablename\ \thetable{} \textendash{} continued from previous page}}}\\
\hline

\endhead

\hline
\multicolumn{2}{r}{\makebox[0pt][r]{\sphinxtablecontinued{continues on next page}}}\\
\endfoot

\endlastfoot

\sphinxAtStartPar
{\hyperref[\detokenize{_autosummary/planetscope.mymodule1.myGlobalFunction:planetscope.mymodule1.myGlobalFunction}]{\sphinxcrossref{\sphinxcode{\sphinxupquote{myGlobalFunction}}}}}
&
\sphinxAtStartPar
This is a global function.
\\
\hline
\end{longtable}\sphinxatlongtableend\end{savenotes}


\subsection{planetscope.mymodule1.myGlobalFunction}
\label{\detokenize{_autosummary/planetscope.mymodule1.myGlobalFunction:planetscope-mymodule1-myglobalfunction}}\label{\detokenize{_autosummary/planetscope.mymodule1.myGlobalFunction::doc}}\index{myGlobalFunction() (in module planetscope.mymodule1)@\spxentry{myGlobalFunction()}\spxextra{in module planetscope.mymodule1}}

\begin{fulllineitems}
\phantomsection\label{\detokenize{_autosummary/planetscope.mymodule1.myGlobalFunction:planetscope.mymodule1.myGlobalFunction}}\pysiglinewithargsret{\sphinxbfcode{\sphinxupquote{myGlobalFunction}}}{\emph{\DUrole{n}{arg1}}, \emph{\DUrole{n}{arg2}}}{}
\sphinxAtStartPar
This is a global function.
\begin{quote}\begin{description}
\item[{Parameters}] \leavevmode\begin{itemize}
\item {} 
\sphinxAtStartPar
\sphinxstyleliteralstrong{\sphinxupquote{arg1}} (\sphinxhref{https://docs.python.org/3/library/stdtypes.html\#str}{\sphinxcode{\sphinxupquote{str}}}) \textendash{} Pass this in first.

\item {} 
\sphinxAtStartPar
\sphinxstyleliteralstrong{\sphinxupquote{arg2}} (\sphinxhref{https://docs.python.org/3/library/functions.html\#bool}{\sphinxcode{\sphinxupquote{bool}}}) \textendash{} Pass this in second.

\end{itemize}

\item[{Return type}] \leavevmode
\sphinxAtStartPar
\sphinxhref{https://docs.python.org/3/library/functions.html\#bool}{\sphinxcode{\sphinxupquote{bool}}}

\item[{Returns}] \leavevmode
\sphinxAtStartPar
Whether to go (\sphinxtitleref{True}) or not go (\sphinxtitleref{False}).

\end{description}\end{quote}

\end{fulllineitems}

\subsubsection*{Classes}


\begin{savenotes}\sphinxatlongtablestart\begin{longtable}[c]{\X{1}{2}\X{1}{2}}
\hline

\endfirsthead

\multicolumn{2}{c}%
{\makebox[0pt]{\sphinxtablecontinued{\tablename\ \thetable{} \textendash{} continued from previous page}}}\\
\hline

\endhead

\hline
\multicolumn{2}{r}{\makebox[0pt][r]{\sphinxtablecontinued{continues on next page}}}\\
\endfoot

\endlastfoot

\sphinxAtStartPar
{\hyperref[\detokenize{_autosummary/planetscope.mymodule1.myClass1:planetscope.mymodule1.myClass1}]{\sphinxcrossref{\sphinxcode{\sphinxupquote{myClass1}}}}}
&
\sphinxAtStartPar
This is a base class.
\\
\hline
\end{longtable}\sphinxatlongtableend\end{savenotes}


\subsection{planetscope.mymodule1.myClass1}
\label{\detokenize{_autosummary/planetscope.mymodule1.myClass1:planetscope-mymodule1-myclass1}}\label{\detokenize{_autosummary/planetscope.mymodule1.myClass1::doc}}\index{myClass1 (class in planetscope.mymodule1)@\spxentry{myClass1}\spxextra{class in planetscope.mymodule1}}

\begin{fulllineitems}
\phantomsection\label{\detokenize{_autosummary/planetscope.mymodule1.myClass1:planetscope.mymodule1.myClass1}}\pysiglinewithargsret{\sphinxbfcode{\sphinxupquote{class\DUrole{w}{  }}}\sphinxbfcode{\sphinxupquote{myClass1}}}{\emph{\DUrole{n}{arg1}}, \emph{\DUrole{n}{arg2}}}{}
\sphinxAtStartPar
Bases: \sphinxhref{https://docs.python.org/3/library/functions.html\#object}{\sphinxcode{\sphinxupquote{object}}}

\sphinxAtStartPar
This is a base class.
\begin{quote}\begin{description}
\item[{Parameters}] \leavevmode\begin{itemize}
\item {} 
\sphinxAtStartPar
\sphinxstyleliteralstrong{\sphinxupquote{arg1}} (\sphinxhref{https://docs.python.org/3/library/typing.html\#typing.Union}{\sphinxcode{\sphinxupquote{Union}}}{[}\sphinxhref{https://docs.python.org/3/library/typing.html\#typing.List}{\sphinxcode{\sphinxupquote{List}}}{[}\sphinxhref{https://docs.python.org/3/library/stdtypes.html\#str}{\sphinxcode{\sphinxupquote{str}}}{]}, \sphinxhref{https://docs.python.org/3/library/typing.html\#typing.Tuple}{\sphinxcode{\sphinxupquote{Tuple}}}{[}\sphinxhref{https://docs.python.org/3/library/functions.html\#int}{\sphinxcode{\sphinxupquote{int}}}, \sphinxhref{https://docs.python.org/3/library/functions.html\#int}{\sphinxcode{\sphinxupquote{int}}}{]}{]}) \textendash{} Pass a \sphinxcode{\sphinxupquote{myTypeAlias}} in first.

\item {} 
\sphinxAtStartPar
\sphinxstyleliteralstrong{\sphinxupquote{arg2}} (\sphinxhref{https://docs.python.org/3/library/stdtypes.html\#str}{\sphinxcode{\sphinxupquote{str}}}) \textendash{} Pass a string in second.

\end{itemize}

\end{description}\end{quote}
\subsubsection*{Methods}


\begin{savenotes}\sphinxatlongtablestart\begin{longtable}[c]{\X{1}{2}\X{1}{2}}
\hline

\endfirsthead

\multicolumn{2}{c}%
{\makebox[0pt]{\sphinxtablecontinued{\tablename\ \thetable{} \textendash{} continued from previous page}}}\\
\hline

\endhead

\hline
\multicolumn{2}{r}{\makebox[0pt][r]{\sphinxtablecontinued{continues on next page}}}\\
\endfoot

\endlastfoot

\sphinxAtStartPar
{\hyperref[\detokenize{_autosummary/planetscope.mymodule1.myClass1:planetscope.mymodule1.myClass1.myMethod1}]{\sphinxcrossref{\sphinxcode{\sphinxupquote{myMethod1}}}}}
&
\sphinxAtStartPar
This is the first public method.
\\
\hline
\sphinxAtStartPar
{\hyperref[\detokenize{_autosummary/planetscope.mymodule1.myClass1:planetscope.mymodule1.myClass1.myMethod2}]{\sphinxcrossref{\sphinxcode{\sphinxupquote{myMethod2}}}}}
&
\sphinxAtStartPar
This is the second public method.
\\
\hline
\end{longtable}\sphinxatlongtableend\end{savenotes}
\subsubsection*{Attributes}


\begin{savenotes}\sphinxatlongtablestart\begin{longtable}[c]{\X{1}{2}\X{1}{2}}
\hline

\endfirsthead

\multicolumn{2}{c}%
{\makebox[0pt]{\sphinxtablecontinued{\tablename\ \thetable{} \textendash{} continued from previous page}}}\\
\hline

\endhead

\hline
\multicolumn{2}{r}{\makebox[0pt][r]{\sphinxtablecontinued{continues on next page}}}\\
\endfoot

\endlastfoot

\sphinxAtStartPar
{\hyperref[\detokenize{_autosummary/planetscope.mymodule1.myClass1:planetscope.mymodule1.myClass1.myAttribute}]{\sphinxcrossref{\sphinxcode{\sphinxupquote{myAttribute}}}}}
&
\sphinxAtStartPar
This is a public attribute.
\\
\hline
\end{longtable}\sphinxatlongtableend\end{savenotes}
\index{\_\_call\_\_() (myClass1 method)@\spxentry{\_\_call\_\_()}\spxextra{myClass1 method}}

\begin{fulllineitems}
\phantomsection\label{\detokenize{_autosummary/planetscope.mymodule1.myClass1:planetscope.mymodule1.myClass1.__call__}}\pysiglinewithargsret{\sphinxbfcode{\sphinxupquote{\_\_call\_\_}}}{\emph{\DUrole{n}{arg}}}{}
\sphinxAtStartPar
This class is callable.
\begin{quote}\begin{description}
\item[{Parameters}] \leavevmode
\sphinxAtStartPar
\sphinxstyleliteralstrong{\sphinxupquote{arg}} (\sphinxhref{https://docs.python.org/3/library/functions.html\#int}{\sphinxcode{\sphinxupquote{int}}}) \textendash{} Pass one of these in.

\item[{Return type}] \leavevmode
\sphinxAtStartPar
\sphinxhref{https://docs.python.org/3/library/stdtypes.html\#str}{\sphinxcode{\sphinxupquote{str}}}

\item[{Returns}] \leavevmode
\sphinxAtStartPar
Get one of these out.

\end{description}\end{quote}

\end{fulllineitems}

\index{myAttribute (myClass1 property)@\spxentry{myAttribute}\spxextra{myClass1 property}}

\begin{fulllineitems}
\phantomsection\label{\detokenize{_autosummary/planetscope.mymodule1.myClass1:planetscope.mymodule1.myClass1.myAttribute}}\pysigline{\sphinxbfcode{\sphinxupquote{property\DUrole{w}{  }}}\sphinxbfcode{\sphinxupquote{myAttribute}}\sphinxbfcode{\sphinxupquote{\DUrole{p}{:}\DUrole{w}{  }\sphinxhref{https://docs.python.org/3/library/functions.html\#int}{int}}}}
\sphinxAtStartPar
This is a public attribute.
\begin{quote}\begin{description}
\item[{Return type}] \leavevmode
\sphinxAtStartPar
\sphinxhref{https://docs.python.org/3/library/functions.html\#int}{\sphinxcode{\sphinxupquote{int}}}

\end{description}\end{quote}

\end{fulllineitems}

\index{myMethod1() (myClass1 method)@\spxentry{myMethod1()}\spxextra{myClass1 method}}

\begin{fulllineitems}
\phantomsection\label{\detokenize{_autosummary/planetscope.mymodule1.myClass1:planetscope.mymodule1.myClass1.myMethod1}}\pysiglinewithargsret{\sphinxbfcode{\sphinxupquote{myMethod1}}}{\emph{\DUrole{n}{arg}}}{}
\sphinxAtStartPar
This is the first public method.
\begin{quote}\begin{description}
\item[{Parameters}] \leavevmode
\sphinxAtStartPar
\sphinxstyleliteralstrong{\sphinxupquote{arg}} (\sphinxstyleliteralemphasis{\sphinxupquote{\textasciitilde{}T}}) \textendash{} Pass a \sphinxcode{\sphinxupquote{myTypeVar}} in.

\item[{Return type}] \leavevmode
\sphinxAtStartPar
\sphinxhref{https://docs.python.org/3/library/stdtypes.html\#str}{\sphinxcode{\sphinxupquote{str}}}

\item[{Returns}] \leavevmode
\sphinxAtStartPar
Get one of these out.

\end{description}\end{quote}

\end{fulllineitems}

\index{myMethod2() (myClass1 method)@\spxentry{myMethod2()}\spxextra{myClass1 method}}

\begin{fulllineitems}
\phantomsection\label{\detokenize{_autosummary/planetscope.mymodule1.myClass1:planetscope.mymodule1.myClass1.myMethod2}}\pysiglinewithargsret{\sphinxbfcode{\sphinxupquote{myMethod2}}}{}{}
\sphinxAtStartPar
This is the second public method.
\begin{quote}\begin{description}
\item[{Return type}] \leavevmode
\sphinxAtStartPar
\sphinxhref{https://docs.python.org/3/library/stdtypes.html\#str}{\sphinxcode{\sphinxupquote{str}}}

\end{description}\end{quote}

\end{fulllineitems}


\end{fulllineitems}

\subsubsection*{Exceptions}


\begin{savenotes}\sphinxatlongtablestart\begin{longtable}[c]{\X{1}{2}\X{1}{2}}
\hline

\endfirsthead

\multicolumn{2}{c}%
{\makebox[0pt]{\sphinxtablecontinued{\tablename\ \thetable{} \textendash{} continued from previous page}}}\\
\hline

\endhead

\hline
\multicolumn{2}{r}{\makebox[0pt][r]{\sphinxtablecontinued{continues on next page}}}\\
\endfoot

\endlastfoot

\sphinxAtStartPar
{\hyperref[\detokenize{_autosummary/planetscope.mymodule1.myError:planetscope.mymodule1.myError}]{\sphinxcrossref{\sphinxcode{\sphinxupquote{myError}}}}}
&
\sphinxAtStartPar
This is a custom error.
\\
\hline
\end{longtable}\sphinxatlongtableend\end{savenotes}


\subsection{planetscope.mymodule1.myError}
\label{\detokenize{_autosummary/planetscope.mymodule1.myError:planetscope-mymodule1-myerror}}\label{\detokenize{_autosummary/planetscope.mymodule1.myError::doc}}\index{myError@\spxentry{myError}}

\begin{fulllineitems}
\phantomsection\label{\detokenize{_autosummary/planetscope.mymodule1.myError:planetscope.mymodule1.myError}}\pysigline{\sphinxbfcode{\sphinxupquote{exception\DUrole{w}{  }}}\sphinxbfcode{\sphinxupquote{myError}}}
\sphinxAtStartPar
This is a custom error.

\end{fulllineitems}



\section{planetscope.mymodule2}
\label{\detokenize{_autosummary/planetscope.mymodule2:module-planetscope.mymodule2}}\label{\detokenize{_autosummary/planetscope.mymodule2:planetscope-mymodule2}}\label{\detokenize{_autosummary/planetscope.mymodule2::doc}}\index{module@\spxentry{module}!planetscope.mymodule2@\spxentry{planetscope.mymodule2}}\index{planetscope.mymodule2@\spxentry{planetscope.mymodule2}!module@\spxentry{module}}
\sphinxAtStartPar
Module containing a class derived from another class.
\subsubsection*{Classes}


\begin{savenotes}\sphinxatlongtablestart\begin{longtable}[c]{\X{1}{2}\X{1}{2}}
\hline

\endfirsthead

\multicolumn{2}{c}%
{\makebox[0pt]{\sphinxtablecontinued{\tablename\ \thetable{} \textendash{} continued from previous page}}}\\
\hline

\endhead

\hline
\multicolumn{2}{r}{\makebox[0pt][r]{\sphinxtablecontinued{continues on next page}}}\\
\endfoot

\endlastfoot

\sphinxAtStartPar
{\hyperref[\detokenize{_autosummary/planetscope.mymodule2.myClass2:planetscope.mymodule2.myClass2}]{\sphinxcrossref{\sphinxcode{\sphinxupquote{myClass2}}}}}
&
\sphinxAtStartPar
Derived class showing members from base class.
\\
\hline
\end{longtable}\sphinxatlongtableend\end{savenotes}


\subsection{planetscope.mymodule2.myClass2}
\label{\detokenize{_autosummary/planetscope.mymodule2.myClass2:planetscope-mymodule2-myclass2}}\label{\detokenize{_autosummary/planetscope.mymodule2.myClass2::doc}}\index{myClass2 (class in planetscope.mymodule2)@\spxentry{myClass2}\spxextra{class in planetscope.mymodule2}}

\begin{fulllineitems}
\phantomsection\label{\detokenize{_autosummary/planetscope.mymodule2.myClass2:planetscope.mymodule2.myClass2}}\pysiglinewithargsret{\sphinxbfcode{\sphinxupquote{class\DUrole{w}{  }}}\sphinxbfcode{\sphinxupquote{myClass2}}}{\emph{\DUrole{n}{arg1}}, \emph{\DUrole{n}{arg2}}}{}
\sphinxAtStartPar
Bases: {\hyperref[\detokenize{_autosummary/planetscope.mymodule1.myClass1:planetscope.mymodule1.myClass1}]{\sphinxcrossref{\sphinxcode{\sphinxupquote{planetscope.mymodule1.myClass1}}}}}

\sphinxAtStartPar
Derived class showing members from base class.
\begin{quote}\begin{description}
\item[{Parameters}] \leavevmode\begin{itemize}
\item {} 
\sphinxAtStartPar
\sphinxstyleliteralstrong{\sphinxupquote{arg1}} (\sphinxhref{https://docs.python.org/3/library/typing.html\#typing.Union}{\sphinxcode{\sphinxupquote{Union}}}{[}\sphinxhref{https://docs.python.org/3/library/typing.html\#typing.List}{\sphinxcode{\sphinxupquote{List}}}{[}\sphinxhref{https://docs.python.org/3/library/stdtypes.html\#str}{\sphinxcode{\sphinxupquote{str}}}{]}, \sphinxhref{https://docs.python.org/3/library/typing.html\#typing.Tuple}{\sphinxcode{\sphinxupquote{Tuple}}}{[}\sphinxhref{https://docs.python.org/3/library/functions.html\#int}{\sphinxcode{\sphinxupquote{int}}}, \sphinxhref{https://docs.python.org/3/library/functions.html\#int}{\sphinxcode{\sphinxupquote{int}}}{]}{]}) \textendash{} Pass a \sphinxcode{\sphinxupquote{myTypeAlias}} in first.

\item {} 
\sphinxAtStartPar
\sphinxstyleliteralstrong{\sphinxupquote{arg2}} (\sphinxhref{https://docs.python.org/3/library/stdtypes.html\#str}{\sphinxcode{\sphinxupquote{str}}}) \textendash{} Pass a string in second.

\end{itemize}

\end{description}\end{quote}
\subsubsection*{Methods}


\begin{savenotes}\sphinxatlongtablestart\begin{longtable}[c]{\X{1}{2}\X{1}{2}}
\hline

\endfirsthead

\multicolumn{2}{c}%
{\makebox[0pt]{\sphinxtablecontinued{\tablename\ \thetable{} \textendash{} continued from previous page}}}\\
\hline

\endhead

\hline
\multicolumn{2}{r}{\makebox[0pt][r]{\sphinxtablecontinued{continues on next page}}}\\
\endfoot

\endlastfoot

\sphinxAtStartPar
{\hyperref[\detokenize{_autosummary/planetscope.mymodule2.myClass2:planetscope.mymodule2.myClass2.myMethod1}]{\sphinxcrossref{\sphinxcode{\sphinxupquote{myMethod1}}}}}
&
\sphinxAtStartPar
This is the first public method.
\\
\hline
\sphinxAtStartPar
{\hyperref[\detokenize{_autosummary/planetscope.mymodule2.myClass2:planetscope.mymodule2.myClass2.myMethod2}]{\sphinxcrossref{\sphinxcode{\sphinxupquote{myMethod2}}}}}
&
\sphinxAtStartPar
This is the second public method.
\\
\hline
\end{longtable}\sphinxatlongtableend\end{savenotes}
\subsubsection*{Attributes}


\begin{savenotes}\sphinxatlongtablestart\begin{longtable}[c]{\X{1}{2}\X{1}{2}}
\hline

\endfirsthead

\multicolumn{2}{c}%
{\makebox[0pt]{\sphinxtablecontinued{\tablename\ \thetable{} \textendash{} continued from previous page}}}\\
\hline

\endhead

\hline
\multicolumn{2}{r}{\makebox[0pt][r]{\sphinxtablecontinued{continues on next page}}}\\
\endfoot

\endlastfoot

\sphinxAtStartPar
{\hyperref[\detokenize{_autosummary/planetscope.mymodule2.myClass2:planetscope.mymodule2.myClass2.myAttribute}]{\sphinxcrossref{\sphinxcode{\sphinxupquote{myAttribute}}}}}
&
\sphinxAtStartPar
This is a public attribute.
\\
\hline
\end{longtable}\sphinxatlongtableend\end{savenotes}
\index{\_\_call\_\_() (myClass2 method)@\spxentry{\_\_call\_\_()}\spxextra{myClass2 method}}

\begin{fulllineitems}
\phantomsection\label{\detokenize{_autosummary/planetscope.mymodule2.myClass2:planetscope.mymodule2.myClass2.__call__}}\pysiglinewithargsret{\sphinxbfcode{\sphinxupquote{\_\_call\_\_}}}{\emph{\DUrole{n}{arg}}}{}
\sphinxAtStartPar
This class is callable.
\begin{quote}\begin{description}
\item[{Parameters}] \leavevmode
\sphinxAtStartPar
\sphinxstyleliteralstrong{\sphinxupquote{arg}} (\sphinxhref{https://docs.python.org/3/library/functions.html\#int}{\sphinxcode{\sphinxupquote{int}}}) \textendash{} Pass one of these in.

\item[{Return type}] \leavevmode
\sphinxAtStartPar
\sphinxhref{https://docs.python.org/3/library/stdtypes.html\#str}{\sphinxcode{\sphinxupquote{str}}}

\item[{Returns}] \leavevmode
\sphinxAtStartPar
Get one of these out.

\end{description}\end{quote}

\end{fulllineitems}

\index{myAttribute (myClass2 property)@\spxentry{myAttribute}\spxextra{myClass2 property}}

\begin{fulllineitems}
\phantomsection\label{\detokenize{_autosummary/planetscope.mymodule2.myClass2:planetscope.mymodule2.myClass2.myAttribute}}\pysigline{\sphinxbfcode{\sphinxupquote{property\DUrole{w}{  }}}\sphinxbfcode{\sphinxupquote{myAttribute}}\sphinxbfcode{\sphinxupquote{\DUrole{p}{:}\DUrole{w}{  }\sphinxhref{https://docs.python.org/3/library/functions.html\#int}{int}}}}
\sphinxAtStartPar
This is a public attribute.
\begin{quote}\begin{description}
\item[{Return type}] \leavevmode
\sphinxAtStartPar
\sphinxhref{https://docs.python.org/3/library/functions.html\#int}{\sphinxcode{\sphinxupquote{int}}}

\end{description}\end{quote}

\end{fulllineitems}

\index{myMethod1() (myClass2 method)@\spxentry{myMethod1()}\spxextra{myClass2 method}}

\begin{fulllineitems}
\phantomsection\label{\detokenize{_autosummary/planetscope.mymodule2.myClass2:planetscope.mymodule2.myClass2.myMethod1}}\pysiglinewithargsret{\sphinxbfcode{\sphinxupquote{myMethod1}}}{\emph{\DUrole{n}{arg}}}{}
\sphinxAtStartPar
This is the first public method.
\begin{quote}\begin{description}
\item[{Parameters}] \leavevmode
\sphinxAtStartPar
\sphinxstyleliteralstrong{\sphinxupquote{arg}} (\sphinxstyleliteralemphasis{\sphinxupquote{\textasciitilde{}T}}) \textendash{} Pass a \sphinxcode{\sphinxupquote{myTypeVar}} in.

\item[{Return type}] \leavevmode
\sphinxAtStartPar
\sphinxhref{https://docs.python.org/3/library/stdtypes.html\#str}{\sphinxcode{\sphinxupquote{str}}}

\item[{Returns}] \leavevmode
\sphinxAtStartPar
Get one of these out.

\end{description}\end{quote}

\end{fulllineitems}

\index{myMethod2() (myClass2 method)@\spxentry{myMethod2()}\spxextra{myClass2 method}}

\begin{fulllineitems}
\phantomsection\label{\detokenize{_autosummary/planetscope.mymodule2.myClass2:planetscope.mymodule2.myClass2.myMethod2}}\pysiglinewithargsret{\sphinxbfcode{\sphinxupquote{myMethod2}}}{}{}
\sphinxAtStartPar
This is the second public method.
\begin{quote}\begin{description}
\item[{Return type}] \leavevmode
\sphinxAtStartPar
\sphinxhref{https://docs.python.org/3/library/stdtypes.html\#str}{\sphinxcode{\sphinxupquote{str}}}

\end{description}\end{quote}

\end{fulllineitems}


\end{fulllineitems}



\section{planetscope.mymodule3a}
\label{\detokenize{_autosummary/planetscope.mymodule3a:module-planetscope.mymodule3a}}\label{\detokenize{_autosummary/planetscope.mymodule3a:planetscope-mymodule3a}}\label{\detokenize{_autosummary/planetscope.mymodule3a::doc}}\index{module@\spxentry{module}!planetscope.mymodule3a@\spxentry{planetscope.mymodule3a}}\index{planetscope.mymodule3a@\spxentry{planetscope.mymodule3a}!module@\spxentry{module}}
\sphinxAtStartPar
Module containing a class derived from another class.
\subsubsection*{Classes}


\begin{savenotes}\sphinxatlongtablestart\begin{longtable}[c]{\X{1}{2}\X{1}{2}}
\hline

\endfirsthead

\multicolumn{2}{c}%
{\makebox[0pt]{\sphinxtablecontinued{\tablename\ \thetable{} \textendash{} continued from previous page}}}\\
\hline

\endhead

\hline
\multicolumn{2}{r}{\makebox[0pt][r]{\sphinxtablecontinued{continues on next page}}}\\
\endfoot

\endlastfoot

\sphinxAtStartPar
{\hyperref[\detokenize{_autosummary/planetscope.mymodule3a.myClass2:planetscope.mymodule3a.myClass2}]{\sphinxcrossref{\sphinxcode{\sphinxupquote{myClass2}}}}}
&
\sphinxAtStartPar
Derived class showing members from base class.
\\
\hline
\end{longtable}\sphinxatlongtableend\end{savenotes}


\subsection{planetscope.mymodule3a.myClass2}
\label{\detokenize{_autosummary/planetscope.mymodule3a.myClass2:planetscope-mymodule3a-myclass2}}\label{\detokenize{_autosummary/planetscope.mymodule3a.myClass2::doc}}\index{myClass2 (class in planetscope.mymodule3a)@\spxentry{myClass2}\spxextra{class in planetscope.mymodule3a}}

\begin{fulllineitems}
\phantomsection\label{\detokenize{_autosummary/planetscope.mymodule3a.myClass2:planetscope.mymodule3a.myClass2}}\pysiglinewithargsret{\sphinxbfcode{\sphinxupquote{class\DUrole{w}{  }}}\sphinxbfcode{\sphinxupquote{myClass2}}}{\emph{\DUrole{n}{arg1}}, \emph{\DUrole{n}{arg2}}}{}
\sphinxAtStartPar
Bases: {\hyperref[\detokenize{_autosummary/planetscope.mymodule1.myClass1:planetscope.mymodule1.myClass1}]{\sphinxcrossref{\sphinxcode{\sphinxupquote{planetscope.mymodule1.myClass1}}}}}

\sphinxAtStartPar
Derived class showing members from base class.
\begin{quote}\begin{description}
\item[{Parameters}] \leavevmode\begin{itemize}
\item {} 
\sphinxAtStartPar
\sphinxstyleliteralstrong{\sphinxupquote{arg1}} (\sphinxhref{https://docs.python.org/3/library/typing.html\#typing.Union}{\sphinxcode{\sphinxupquote{Union}}}{[}\sphinxhref{https://docs.python.org/3/library/typing.html\#typing.List}{\sphinxcode{\sphinxupquote{List}}}{[}\sphinxhref{https://docs.python.org/3/library/stdtypes.html\#str}{\sphinxcode{\sphinxupquote{str}}}{]}, \sphinxhref{https://docs.python.org/3/library/typing.html\#typing.Tuple}{\sphinxcode{\sphinxupquote{Tuple}}}{[}\sphinxhref{https://docs.python.org/3/library/functions.html\#int}{\sphinxcode{\sphinxupquote{int}}}, \sphinxhref{https://docs.python.org/3/library/functions.html\#int}{\sphinxcode{\sphinxupquote{int}}}{]}{]}) \textendash{} Pass a \sphinxcode{\sphinxupquote{myTypeAlias}} in first.

\item {} 
\sphinxAtStartPar
\sphinxstyleliteralstrong{\sphinxupquote{arg2}} (\sphinxhref{https://docs.python.org/3/library/stdtypes.html\#str}{\sphinxcode{\sphinxupquote{str}}}) \textendash{} Pass a string in second.

\end{itemize}

\end{description}\end{quote}
\subsubsection*{Methods}


\begin{savenotes}\sphinxatlongtablestart\begin{longtable}[c]{\X{1}{2}\X{1}{2}}
\hline

\endfirsthead

\multicolumn{2}{c}%
{\makebox[0pt]{\sphinxtablecontinued{\tablename\ \thetable{} \textendash{} continued from previous page}}}\\
\hline

\endhead

\hline
\multicolumn{2}{r}{\makebox[0pt][r]{\sphinxtablecontinued{continues on next page}}}\\
\endfoot

\endlastfoot

\sphinxAtStartPar
{\hyperref[\detokenize{_autosummary/planetscope.mymodule3a.myClass2:planetscope.mymodule3a.myClass2.myMethod1}]{\sphinxcrossref{\sphinxcode{\sphinxupquote{myMethod1}}}}}
&
\sphinxAtStartPar
This is the first public method.
\\
\hline
\sphinxAtStartPar
{\hyperref[\detokenize{_autosummary/planetscope.mymodule3a.myClass2:planetscope.mymodule3a.myClass2.myMethod2}]{\sphinxcrossref{\sphinxcode{\sphinxupquote{myMethod2}}}}}
&
\sphinxAtStartPar
This is the second public method.
\\
\hline
\end{longtable}\sphinxatlongtableend\end{savenotes}
\subsubsection*{Attributes}


\begin{savenotes}\sphinxatlongtablestart\begin{longtable}[c]{\X{1}{2}\X{1}{2}}
\hline

\endfirsthead

\multicolumn{2}{c}%
{\makebox[0pt]{\sphinxtablecontinued{\tablename\ \thetable{} \textendash{} continued from previous page}}}\\
\hline

\endhead

\hline
\multicolumn{2}{r}{\makebox[0pt][r]{\sphinxtablecontinued{continues on next page}}}\\
\endfoot

\endlastfoot

\sphinxAtStartPar
{\hyperref[\detokenize{_autosummary/planetscope.mymodule3a.myClass2:planetscope.mymodule3a.myClass2.myAttribute}]{\sphinxcrossref{\sphinxcode{\sphinxupquote{myAttribute}}}}}
&
\sphinxAtStartPar
This is a public attribute.
\\
\hline
\end{longtable}\sphinxatlongtableend\end{savenotes}
\index{\_\_call\_\_() (myClass2 method)@\spxentry{\_\_call\_\_()}\spxextra{myClass2 method}}

\begin{fulllineitems}
\phantomsection\label{\detokenize{_autosummary/planetscope.mymodule3a.myClass2:planetscope.mymodule3a.myClass2.__call__}}\pysiglinewithargsret{\sphinxbfcode{\sphinxupquote{\_\_call\_\_}}}{\emph{\DUrole{n}{arg}}}{}
\sphinxAtStartPar
This class is callable.
\begin{quote}\begin{description}
\item[{Parameters}] \leavevmode
\sphinxAtStartPar
\sphinxstyleliteralstrong{\sphinxupquote{arg}} (\sphinxhref{https://docs.python.org/3/library/functions.html\#int}{\sphinxcode{\sphinxupquote{int}}}) \textendash{} Pass one of these in.

\item[{Return type}] \leavevmode
\sphinxAtStartPar
\sphinxhref{https://docs.python.org/3/library/stdtypes.html\#str}{\sphinxcode{\sphinxupquote{str}}}

\item[{Returns}] \leavevmode
\sphinxAtStartPar
Get one of these out.

\end{description}\end{quote}

\end{fulllineitems}

\index{myAttribute (myClass2 property)@\spxentry{myAttribute}\spxextra{myClass2 property}}

\begin{fulllineitems}
\phantomsection\label{\detokenize{_autosummary/planetscope.mymodule3a.myClass2:planetscope.mymodule3a.myClass2.myAttribute}}\pysigline{\sphinxbfcode{\sphinxupquote{property\DUrole{w}{  }}}\sphinxbfcode{\sphinxupquote{myAttribute}}\sphinxbfcode{\sphinxupquote{\DUrole{p}{:}\DUrole{w}{  }\sphinxhref{https://docs.python.org/3/library/functions.html\#int}{int}}}}
\sphinxAtStartPar
This is a public attribute.
\begin{quote}\begin{description}
\item[{Return type}] \leavevmode
\sphinxAtStartPar
\sphinxhref{https://docs.python.org/3/library/functions.html\#int}{\sphinxcode{\sphinxupquote{int}}}

\end{description}\end{quote}

\end{fulllineitems}

\index{myMethod1() (myClass2 method)@\spxentry{myMethod1()}\spxextra{myClass2 method}}

\begin{fulllineitems}
\phantomsection\label{\detokenize{_autosummary/planetscope.mymodule3a.myClass2:planetscope.mymodule3a.myClass2.myMethod1}}\pysiglinewithargsret{\sphinxbfcode{\sphinxupquote{myMethod1}}}{\emph{\DUrole{n}{arg}}}{}
\sphinxAtStartPar
This is the first public method.
\begin{quote}\begin{description}
\item[{Parameters}] \leavevmode
\sphinxAtStartPar
\sphinxstyleliteralstrong{\sphinxupquote{arg}} (\sphinxstyleliteralemphasis{\sphinxupquote{\textasciitilde{}T}}) \textendash{} Pass a \sphinxcode{\sphinxupquote{myTypeVar}} in.

\item[{Return type}] \leavevmode
\sphinxAtStartPar
\sphinxhref{https://docs.python.org/3/library/stdtypes.html\#str}{\sphinxcode{\sphinxupquote{str}}}

\item[{Returns}] \leavevmode
\sphinxAtStartPar
Get one of these out.

\end{description}\end{quote}

\end{fulllineitems}

\index{myMethod2() (myClass2 method)@\spxentry{myMethod2()}\spxextra{myClass2 method}}

\begin{fulllineitems}
\phantomsection\label{\detokenize{_autosummary/planetscope.mymodule3a.myClass2:planetscope.mymodule3a.myClass2.myMethod2}}\pysiglinewithargsret{\sphinxbfcode{\sphinxupquote{myMethod2}}}{}{}
\sphinxAtStartPar
This is the second public method.
\begin{quote}\begin{description}
\item[{Return type}] \leavevmode
\sphinxAtStartPar
\sphinxhref{https://docs.python.org/3/library/stdtypes.html\#str}{\sphinxcode{\sphinxupquote{str}}}

\end{description}\end{quote}

\end{fulllineitems}


\end{fulllineitems}



\section{planetscope.mysubpackage}
\label{\detokenize{_autosummary/planetscope.mysubpackage:module-planetscope.mysubpackage}}\label{\detokenize{_autosummary/planetscope.mysubpackage:planetscope-mysubpackage}}\label{\detokenize{_autosummary/planetscope.mysubpackage::doc}}\index{module@\spxentry{module}!planetscope.mysubpackage@\spxentry{planetscope.mysubpackage}}\index{planetscope.mysubpackage@\spxentry{planetscope.mysubpackage}!module@\spxentry{module}}
\sphinxAtStartPar
Package demonstrating one level of nesting.


\begin{savenotes}\sphinxatlongtablestart\begin{longtable}[c]{\X{1}{2}\X{1}{2}}
\hline

\endfirsthead

\multicolumn{2}{c}%
{\makebox[0pt]{\sphinxtablecontinued{\tablename\ \thetable{} \textendash{} continued from previous page}}}\\
\hline

\endhead

\hline
\multicolumn{2}{r}{\makebox[0pt][r]{\sphinxtablecontinued{continues on next page}}}\\
\endfoot

\endlastfoot

\sphinxAtStartPar
{\hyperref[\detokenize{_autosummary/planetscope.mysubpackage.mymodule3:module-planetscope.mysubpackage.mymodule3}]{\sphinxcrossref{\sphinxcode{\sphinxupquote{planetscope.mysubpackage.mymodule3}}}}}
&
\sphinxAtStartPar
Module containing a third class.
\\
\hline
\sphinxAtStartPar
{\hyperref[\detokenize{_autosummary/planetscope.mysubpackage.mysubsubpackage:module-planetscope.mysubpackage.mysubsubpackage}]{\sphinxcrossref{\sphinxcode{\sphinxupquote{planetscope.mysubpackage.mysubsubpackage}}}}}
&
\sphinxAtStartPar
Package demonstrating two levels of nesting.
\\
\hline
\end{longtable}\sphinxatlongtableend\end{savenotes}


\subsection{planetscope.mysubpackage.mymodule3}
\label{\detokenize{_autosummary/planetscope.mysubpackage.mymodule3:module-planetscope.mysubpackage.mymodule3}}\label{\detokenize{_autosummary/planetscope.mysubpackage.mymodule3:planetscope-mysubpackage-mymodule3}}\label{\detokenize{_autosummary/planetscope.mysubpackage.mymodule3::doc}}\index{module@\spxentry{module}!planetscope.mysubpackage.mymodule3@\spxentry{planetscope.mysubpackage.mymodule3}}\index{planetscope.mysubpackage.mymodule3@\spxentry{planetscope.mysubpackage.mymodule3}!module@\spxentry{module}}
\sphinxAtStartPar
Module containing a third class.
\subsubsection*{Classes}


\begin{savenotes}\sphinxatlongtablestart\begin{longtable}[c]{\X{1}{2}\X{1}{2}}
\hline

\endfirsthead

\multicolumn{2}{c}%
{\makebox[0pt]{\sphinxtablecontinued{\tablename\ \thetable{} \textendash{} continued from previous page}}}\\
\hline

\endhead

\hline
\multicolumn{2}{r}{\makebox[0pt][r]{\sphinxtablecontinued{continues on next page}}}\\
\endfoot

\endlastfoot

\sphinxAtStartPar
{\hyperref[\detokenize{_autosummary/planetscope.mysubpackage.mymodule3.myClass3:planetscope.mysubpackage.mymodule3.myClass3}]{\sphinxcrossref{\sphinxcode{\sphinxupquote{myClass3}}}}}
&
\sphinxAtStartPar
This is the third class.
\\
\hline
\end{longtable}\sphinxatlongtableend\end{savenotes}


\subsubsection{planetscope.mysubpackage.mymodule3.myClass3}
\label{\detokenize{_autosummary/planetscope.mysubpackage.mymodule3.myClass3:planetscope-mysubpackage-mymodule3-myclass3}}\label{\detokenize{_autosummary/planetscope.mysubpackage.mymodule3.myClass3::doc}}\index{myClass3 (class in planetscope.mysubpackage.mymodule3)@\spxentry{myClass3}\spxextra{class in planetscope.mysubpackage.mymodule3}}

\begin{fulllineitems}
\phantomsection\label{\detokenize{_autosummary/planetscope.mysubpackage.mymodule3.myClass3:planetscope.mysubpackage.mymodule3.myClass3}}\pysiglinewithargsret{\sphinxbfcode{\sphinxupquote{class\DUrole{w}{  }}}\sphinxbfcode{\sphinxupquote{myClass3}}}{\emph{\DUrole{n}{arg1}}, \emph{\DUrole{n}{arg2}}}{}
\sphinxAtStartPar
Bases: {\hyperref[\detokenize{_autosummary/planetscope.mymodule1.myClass1:planetscope.mymodule1.myClass1}]{\sphinxcrossref{\sphinxcode{\sphinxupquote{planetscope.mymodule1.myClass1}}}}}

\sphinxAtStartPar
This is the third class.
\begin{quote}\begin{description}
\item[{Parameters}] \leavevmode\begin{itemize}
\item {} 
\sphinxAtStartPar
\sphinxstyleliteralstrong{\sphinxupquote{arg1}} (\sphinxhref{https://docs.python.org/3/library/typing.html\#typing.Union}{\sphinxcode{\sphinxupquote{Union}}}{[}\sphinxhref{https://docs.python.org/3/library/typing.html\#typing.List}{\sphinxcode{\sphinxupquote{List}}}{[}\sphinxhref{https://docs.python.org/3/library/stdtypes.html\#str}{\sphinxcode{\sphinxupquote{str}}}{]}, \sphinxhref{https://docs.python.org/3/library/typing.html\#typing.Tuple}{\sphinxcode{\sphinxupquote{Tuple}}}{[}\sphinxhref{https://docs.python.org/3/library/functions.html\#int}{\sphinxcode{\sphinxupquote{int}}}, \sphinxhref{https://docs.python.org/3/library/functions.html\#int}{\sphinxcode{\sphinxupquote{int}}}{]}{]}) \textendash{} Pass a \sphinxcode{\sphinxupquote{myTypeAlias}} in first.

\item {} 
\sphinxAtStartPar
\sphinxstyleliteralstrong{\sphinxupquote{arg2}} (\sphinxhref{https://docs.python.org/3/library/stdtypes.html\#str}{\sphinxcode{\sphinxupquote{str}}}) \textendash{} Pass a string in second.

\end{itemize}

\end{description}\end{quote}
\subsubsection*{Methods}


\begin{savenotes}\sphinxatlongtablestart\begin{longtable}[c]{\X{1}{2}\X{1}{2}}
\hline

\endfirsthead

\multicolumn{2}{c}%
{\makebox[0pt]{\sphinxtablecontinued{\tablename\ \thetable{} \textendash{} continued from previous page}}}\\
\hline

\endhead

\hline
\multicolumn{2}{r}{\makebox[0pt][r]{\sphinxtablecontinued{continues on next page}}}\\
\endfoot

\endlastfoot

\sphinxAtStartPar
{\hyperref[\detokenize{_autosummary/planetscope.mysubpackage.mymodule3.myClass3:planetscope.mysubpackage.mymodule3.myClass3.myMethod1}]{\sphinxcrossref{\sphinxcode{\sphinxupquote{myMethod1}}}}}
&
\sphinxAtStartPar
This is the first public method.
\\
\hline
\sphinxAtStartPar
{\hyperref[\detokenize{_autosummary/planetscope.mysubpackage.mymodule3.myClass3:planetscope.mysubpackage.mymodule3.myClass3.myMethod2}]{\sphinxcrossref{\sphinxcode{\sphinxupquote{myMethod2}}}}}
&
\sphinxAtStartPar
This is the second public method.
\\
\hline
\end{longtable}\sphinxatlongtableend\end{savenotes}
\subsubsection*{Attributes}


\begin{savenotes}\sphinxatlongtablestart\begin{longtable}[c]{\X{1}{2}\X{1}{2}}
\hline

\endfirsthead

\multicolumn{2}{c}%
{\makebox[0pt]{\sphinxtablecontinued{\tablename\ \thetable{} \textendash{} continued from previous page}}}\\
\hline

\endhead

\hline
\multicolumn{2}{r}{\makebox[0pt][r]{\sphinxtablecontinued{continues on next page}}}\\
\endfoot

\endlastfoot

\sphinxAtStartPar
{\hyperref[\detokenize{_autosummary/planetscope.mysubpackage.mymodule3.myClass3:planetscope.mysubpackage.mymodule3.myClass3.myAttribute}]{\sphinxcrossref{\sphinxcode{\sphinxupquote{myAttribute}}}}}
&
\sphinxAtStartPar
This is a public attribute.
\\
\hline
\end{longtable}\sphinxatlongtableend\end{savenotes}
\index{\_\_call\_\_() (myClass3 method)@\spxentry{\_\_call\_\_()}\spxextra{myClass3 method}}

\begin{fulllineitems}
\phantomsection\label{\detokenize{_autosummary/planetscope.mysubpackage.mymodule3.myClass3:planetscope.mysubpackage.mymodule3.myClass3.__call__}}\pysiglinewithargsret{\sphinxbfcode{\sphinxupquote{\_\_call\_\_}}}{\emph{\DUrole{n}{arg}}}{}
\sphinxAtStartPar
This class is callable.
\begin{quote}\begin{description}
\item[{Parameters}] \leavevmode
\sphinxAtStartPar
\sphinxstyleliteralstrong{\sphinxupquote{arg}} (\sphinxhref{https://docs.python.org/3/library/functions.html\#int}{\sphinxcode{\sphinxupquote{int}}}) \textendash{} Pass one of these in.

\item[{Return type}] \leavevmode
\sphinxAtStartPar
\sphinxhref{https://docs.python.org/3/library/stdtypes.html\#str}{\sphinxcode{\sphinxupquote{str}}}

\item[{Returns}] \leavevmode
\sphinxAtStartPar
Get one of these out.

\end{description}\end{quote}

\end{fulllineitems}

\index{myAttribute (myClass3 property)@\spxentry{myAttribute}\spxextra{myClass3 property}}

\begin{fulllineitems}
\phantomsection\label{\detokenize{_autosummary/planetscope.mysubpackage.mymodule3.myClass3:planetscope.mysubpackage.mymodule3.myClass3.myAttribute}}\pysigline{\sphinxbfcode{\sphinxupquote{property\DUrole{w}{  }}}\sphinxbfcode{\sphinxupquote{myAttribute}}\sphinxbfcode{\sphinxupquote{\DUrole{p}{:}\DUrole{w}{  }\sphinxhref{https://docs.python.org/3/library/functions.html\#int}{int}}}}
\sphinxAtStartPar
This is a public attribute.
\begin{quote}\begin{description}
\item[{Return type}] \leavevmode
\sphinxAtStartPar
\sphinxhref{https://docs.python.org/3/library/functions.html\#int}{\sphinxcode{\sphinxupquote{int}}}

\end{description}\end{quote}

\end{fulllineitems}

\index{myMethod1() (myClass3 method)@\spxentry{myMethod1()}\spxextra{myClass3 method}}

\begin{fulllineitems}
\phantomsection\label{\detokenize{_autosummary/planetscope.mysubpackage.mymodule3.myClass3:planetscope.mysubpackage.mymodule3.myClass3.myMethod1}}\pysiglinewithargsret{\sphinxbfcode{\sphinxupquote{myMethod1}}}{\emph{\DUrole{n}{arg}}}{}
\sphinxAtStartPar
This is the first public method.
\begin{quote}\begin{description}
\item[{Parameters}] \leavevmode
\sphinxAtStartPar
\sphinxstyleliteralstrong{\sphinxupquote{arg}} (\sphinxstyleliteralemphasis{\sphinxupquote{\textasciitilde{}T}}) \textendash{} Pass a \sphinxcode{\sphinxupquote{myTypeVar}} in.

\item[{Return type}] \leavevmode
\sphinxAtStartPar
\sphinxhref{https://docs.python.org/3/library/stdtypes.html\#str}{\sphinxcode{\sphinxupquote{str}}}

\item[{Returns}] \leavevmode
\sphinxAtStartPar
Get one of these out.

\end{description}\end{quote}

\end{fulllineitems}

\index{myMethod2() (myClass3 method)@\spxentry{myMethod2()}\spxextra{myClass3 method}}

\begin{fulllineitems}
\phantomsection\label{\detokenize{_autosummary/planetscope.mysubpackage.mymodule3.myClass3:planetscope.mysubpackage.mymodule3.myClass3.myMethod2}}\pysiglinewithargsret{\sphinxbfcode{\sphinxupquote{myMethod2}}}{}{}
\sphinxAtStartPar
This is the second public method.
\begin{quote}\begin{description}
\item[{Return type}] \leavevmode
\sphinxAtStartPar
\sphinxhref{https://docs.python.org/3/library/stdtypes.html\#str}{\sphinxcode{\sphinxupquote{str}}}

\end{description}\end{quote}

\end{fulllineitems}


\end{fulllineitems}



\subsection{planetscope.mysubpackage.mysubsubpackage}
\label{\detokenize{_autosummary/planetscope.mysubpackage.mysubsubpackage:module-planetscope.mysubpackage.mysubsubpackage}}\label{\detokenize{_autosummary/planetscope.mysubpackage.mysubsubpackage:planetscope-mysubpackage-mysubsubpackage}}\label{\detokenize{_autosummary/planetscope.mysubpackage.mysubsubpackage::doc}}\index{module@\spxentry{module}!planetscope.mysubpackage.mysubsubpackage@\spxentry{planetscope.mysubpackage.mysubsubpackage}}\index{planetscope.mysubpackage.mysubsubpackage@\spxentry{planetscope.mysubpackage.mysubsubpackage}!module@\spxentry{module}}
\sphinxAtStartPar
Package demonstrating two levels of nesting.


\begin{savenotes}\sphinxatlongtablestart\begin{longtable}[c]{\X{1}{2}\X{1}{2}}
\hline

\endfirsthead

\multicolumn{2}{c}%
{\makebox[0pt]{\sphinxtablecontinued{\tablename\ \thetable{} \textendash{} continued from previous page}}}\\
\hline

\endhead

\hline
\multicolumn{2}{r}{\makebox[0pt][r]{\sphinxtablecontinued{continues on next page}}}\\
\endfoot

\endlastfoot

\sphinxAtStartPar
{\hyperref[\detokenize{_autosummary/planetscope.mysubpackage.mysubsubpackage.mymodule4:module-planetscope.mysubpackage.mysubsubpackage.mymodule4}]{\sphinxcrossref{\sphinxcode{\sphinxupquote{planetscope.mysubpackage.mysubsubpackage.mymodule4}}}}}
&
\sphinxAtStartPar
Module containing a fourth class.
\\
\hline
\end{longtable}\sphinxatlongtableend\end{savenotes}


\subsubsection{planetscope.mysubpackage.mysubsubpackage.mymodule4}
\label{\detokenize{_autosummary/planetscope.mysubpackage.mysubsubpackage.mymodule4:module-planetscope.mysubpackage.mysubsubpackage.mymodule4}}\label{\detokenize{_autosummary/planetscope.mysubpackage.mysubsubpackage.mymodule4:planetscope-mysubpackage-mysubsubpackage-mymodule4}}\label{\detokenize{_autosummary/planetscope.mysubpackage.mysubsubpackage.mymodule4::doc}}\index{module@\spxentry{module}!planetscope.mysubpackage.mysubsubpackage.mymodule4@\spxentry{planetscope.mysubpackage.mysubsubpackage.mymodule4}}\index{planetscope.mysubpackage.mysubsubpackage.mymodule4@\spxentry{planetscope.mysubpackage.mysubsubpackage.mymodule4}!module@\spxentry{module}}
\sphinxAtStartPar
Module containing a fourth class.
\subsubsection*{Classes}


\begin{savenotes}\sphinxatlongtablestart\begin{longtable}[c]{\X{1}{2}\X{1}{2}}
\hline

\endfirsthead

\multicolumn{2}{c}%
{\makebox[0pt]{\sphinxtablecontinued{\tablename\ \thetable{} \textendash{} continued from previous page}}}\\
\hline

\endhead

\hline
\multicolumn{2}{r}{\makebox[0pt][r]{\sphinxtablecontinued{continues on next page}}}\\
\endfoot

\endlastfoot

\sphinxAtStartPar
{\hyperref[\detokenize{_autosummary/planetscope.mysubpackage.mysubsubpackage.mymodule4.myClass4:planetscope.mysubpackage.mysubsubpackage.mymodule4.myClass4}]{\sphinxcrossref{\sphinxcode{\sphinxupquote{myClass4}}}}}
&
\sphinxAtStartPar
This is the fourth class.
\\
\hline
\end{longtable}\sphinxatlongtableend\end{savenotes}


\paragraph{planetscope.mysubpackage.mysubsubpackage.mymodule4.myClass4}
\label{\detokenize{_autosummary/planetscope.mysubpackage.mysubsubpackage.mymodule4.myClass4:planetscope-mysubpackage-mysubsubpackage-mymodule4-myclass4}}\label{\detokenize{_autosummary/planetscope.mysubpackage.mysubsubpackage.mymodule4.myClass4::doc}}\index{myClass4 (class in planetscope.mysubpackage.mysubsubpackage.mymodule4)@\spxentry{myClass4}\spxextra{class in planetscope.mysubpackage.mysubsubpackage.mymodule4}}

\begin{fulllineitems}
\phantomsection\label{\detokenize{_autosummary/planetscope.mysubpackage.mysubsubpackage.mymodule4.myClass4:planetscope.mysubpackage.mysubsubpackage.mymodule4.myClass4}}\pysiglinewithargsret{\sphinxbfcode{\sphinxupquote{class\DUrole{w}{  }}}\sphinxbfcode{\sphinxupquote{myClass4}}}{\emph{\DUrole{n}{arg1}}, \emph{\DUrole{n}{arg2}}}{}
\sphinxAtStartPar
Bases: {\hyperref[\detokenize{_autosummary/planetscope.mymodule1.myClass1:planetscope.mymodule1.myClass1}]{\sphinxcrossref{\sphinxcode{\sphinxupquote{planetscope.mymodule1.myClass1}}}}}

\sphinxAtStartPar
This is the fourth class.
\begin{quote}\begin{description}
\item[{Parameters}] \leavevmode\begin{itemize}
\item {} 
\sphinxAtStartPar
\sphinxstyleliteralstrong{\sphinxupquote{arg1}} (\sphinxhref{https://docs.python.org/3/library/typing.html\#typing.Union}{\sphinxcode{\sphinxupquote{Union}}}{[}\sphinxhref{https://docs.python.org/3/library/typing.html\#typing.List}{\sphinxcode{\sphinxupquote{List}}}{[}\sphinxhref{https://docs.python.org/3/library/stdtypes.html\#str}{\sphinxcode{\sphinxupquote{str}}}{]}, \sphinxhref{https://docs.python.org/3/library/typing.html\#typing.Tuple}{\sphinxcode{\sphinxupquote{Tuple}}}{[}\sphinxhref{https://docs.python.org/3/library/functions.html\#int}{\sphinxcode{\sphinxupquote{int}}}, \sphinxhref{https://docs.python.org/3/library/functions.html\#int}{\sphinxcode{\sphinxupquote{int}}}{]}{]}) \textendash{} Pass a \sphinxcode{\sphinxupquote{myTypeAlias}} in first.

\item {} 
\sphinxAtStartPar
\sphinxstyleliteralstrong{\sphinxupquote{arg2}} (\sphinxhref{https://docs.python.org/3/library/stdtypes.html\#str}{\sphinxcode{\sphinxupquote{str}}}) \textendash{} Pass a string in second.

\end{itemize}

\end{description}\end{quote}
\subsubsection*{Methods}


\begin{savenotes}\sphinxatlongtablestart\begin{longtable}[c]{\X{1}{2}\X{1}{2}}
\hline

\endfirsthead

\multicolumn{2}{c}%
{\makebox[0pt]{\sphinxtablecontinued{\tablename\ \thetable{} \textendash{} continued from previous page}}}\\
\hline

\endhead

\hline
\multicolumn{2}{r}{\makebox[0pt][r]{\sphinxtablecontinued{continues on next page}}}\\
\endfoot

\endlastfoot

\sphinxAtStartPar
{\hyperref[\detokenize{_autosummary/planetscope.mysubpackage.mysubsubpackage.mymodule4.myClass4:planetscope.mysubpackage.mysubsubpackage.mymodule4.myClass4.myMethod1}]{\sphinxcrossref{\sphinxcode{\sphinxupquote{myMethod1}}}}}
&
\sphinxAtStartPar
This is the first public method.
\\
\hline
\sphinxAtStartPar
{\hyperref[\detokenize{_autosummary/planetscope.mysubpackage.mysubsubpackage.mymodule4.myClass4:planetscope.mysubpackage.mysubsubpackage.mymodule4.myClass4.myMethod2}]{\sphinxcrossref{\sphinxcode{\sphinxupquote{myMethod2}}}}}
&
\sphinxAtStartPar
This is the second public method.
\\
\hline
\end{longtable}\sphinxatlongtableend\end{savenotes}
\subsubsection*{Attributes}


\begin{savenotes}\sphinxatlongtablestart\begin{longtable}[c]{\X{1}{2}\X{1}{2}}
\hline

\endfirsthead

\multicolumn{2}{c}%
{\makebox[0pt]{\sphinxtablecontinued{\tablename\ \thetable{} \textendash{} continued from previous page}}}\\
\hline

\endhead

\hline
\multicolumn{2}{r}{\makebox[0pt][r]{\sphinxtablecontinued{continues on next page}}}\\
\endfoot

\endlastfoot

\sphinxAtStartPar
{\hyperref[\detokenize{_autosummary/planetscope.mysubpackage.mysubsubpackage.mymodule4.myClass4:planetscope.mysubpackage.mysubsubpackage.mymodule4.myClass4.myAttribute}]{\sphinxcrossref{\sphinxcode{\sphinxupquote{myAttribute}}}}}
&
\sphinxAtStartPar
This is a public attribute.
\\
\hline
\end{longtable}\sphinxatlongtableend\end{savenotes}
\index{\_\_call\_\_() (myClass4 method)@\spxentry{\_\_call\_\_()}\spxextra{myClass4 method}}

\begin{fulllineitems}
\phantomsection\label{\detokenize{_autosummary/planetscope.mysubpackage.mysubsubpackage.mymodule4.myClass4:planetscope.mysubpackage.mysubsubpackage.mymodule4.myClass4.__call__}}\pysiglinewithargsret{\sphinxbfcode{\sphinxupquote{\_\_call\_\_}}}{\emph{\DUrole{n}{arg}}}{}
\sphinxAtStartPar
This class is callable.
\begin{quote}\begin{description}
\item[{Parameters}] \leavevmode
\sphinxAtStartPar
\sphinxstyleliteralstrong{\sphinxupquote{arg}} (\sphinxhref{https://docs.python.org/3/library/functions.html\#int}{\sphinxcode{\sphinxupquote{int}}}) \textendash{} Pass one of these in.

\item[{Return type}] \leavevmode
\sphinxAtStartPar
\sphinxhref{https://docs.python.org/3/library/stdtypes.html\#str}{\sphinxcode{\sphinxupquote{str}}}

\item[{Returns}] \leavevmode
\sphinxAtStartPar
Get one of these out.

\end{description}\end{quote}

\end{fulllineitems}

\index{myAttribute (myClass4 property)@\spxentry{myAttribute}\spxextra{myClass4 property}}

\begin{fulllineitems}
\phantomsection\label{\detokenize{_autosummary/planetscope.mysubpackage.mysubsubpackage.mymodule4.myClass4:planetscope.mysubpackage.mysubsubpackage.mymodule4.myClass4.myAttribute}}\pysigline{\sphinxbfcode{\sphinxupquote{property\DUrole{w}{  }}}\sphinxbfcode{\sphinxupquote{myAttribute}}\sphinxbfcode{\sphinxupquote{\DUrole{p}{:}\DUrole{w}{  }\sphinxhref{https://docs.python.org/3/library/functions.html\#int}{int}}}}
\sphinxAtStartPar
This is a public attribute.
\begin{quote}\begin{description}
\item[{Return type}] \leavevmode
\sphinxAtStartPar
\sphinxhref{https://docs.python.org/3/library/functions.html\#int}{\sphinxcode{\sphinxupquote{int}}}

\end{description}\end{quote}

\end{fulllineitems}

\index{myMethod1() (myClass4 method)@\spxentry{myMethod1()}\spxextra{myClass4 method}}

\begin{fulllineitems}
\phantomsection\label{\detokenize{_autosummary/planetscope.mysubpackage.mysubsubpackage.mymodule4.myClass4:planetscope.mysubpackage.mysubsubpackage.mymodule4.myClass4.myMethod1}}\pysiglinewithargsret{\sphinxbfcode{\sphinxupquote{myMethod1}}}{\emph{\DUrole{n}{arg}}}{}
\sphinxAtStartPar
This is the first public method.
\begin{quote}\begin{description}
\item[{Parameters}] \leavevmode
\sphinxAtStartPar
\sphinxstyleliteralstrong{\sphinxupquote{arg}} (\sphinxstyleliteralemphasis{\sphinxupquote{\textasciitilde{}T}}) \textendash{} Pass a \sphinxcode{\sphinxupquote{myTypeVar}} in.

\item[{Return type}] \leavevmode
\sphinxAtStartPar
\sphinxhref{https://docs.python.org/3/library/stdtypes.html\#str}{\sphinxcode{\sphinxupquote{str}}}

\item[{Returns}] \leavevmode
\sphinxAtStartPar
Get one of these out.

\end{description}\end{quote}

\end{fulllineitems}

\index{myMethod2() (myClass4 method)@\spxentry{myMethod2()}\spxextra{myClass4 method}}

\begin{fulllineitems}
\phantomsection\label{\detokenize{_autosummary/planetscope.mysubpackage.mysubsubpackage.mymodule4.myClass4:planetscope.mysubpackage.mysubsubpackage.mymodule4.myClass4.myMethod2}}\pysiglinewithargsret{\sphinxbfcode{\sphinxupquote{myMethod2}}}{}{}
\sphinxAtStartPar
This is the second public method.
\begin{quote}\begin{description}
\item[{Return type}] \leavevmode
\sphinxAtStartPar
\sphinxhref{https://docs.python.org/3/library/stdtypes.html\#str}{\sphinxcode{\sphinxupquote{str}}}

\end{description}\end{quote}

\end{fulllineitems}


\end{fulllineitems}



\chapter{Support}
\label{\detokenize{supportfaqs:support}}\label{\detokenize{supportfaqs::doc}}
\sphinxAtStartPar
If you have any questions or encounter any issues while using the planetscope package, there are several ways you can get help:
\begin{itemize}
\item {} 
\sphinxAtStartPar
Consult the documentation: The documentation for the planetscope package should provide detailed information on how to use the package and troubleshoot common issues. You can find the documentation at \sphinxurl{https://github.com/NASA-IMPACT/planet\_utils/}.

\item {} 
\sphinxAtStartPar
Check the FAQs: We have compiled a list of common questions and issues that users may encounter while using the planetscope package. You can find the FAQs at \sphinxurl{https://docs.planetscope.org/faqs/}.

\item {} 
\sphinxAtStartPar
File an issue on GitHub: If you are unable to find a solution to your issue in the documentation or FAQs, you can file an issue on the package’s GitHub repository at \sphinxurl{https://github.com/NASA-IMPACT/planet\_utils/issues}. When filing an issue, please include as much information as possible, such as the version of the package you are using, the Python version you are using, and any error messages you are seeing.

\item {} 
\sphinxAtStartPar
Contact the maintainers: If you have a question or issue that is not covered in the documentation or FAQs, or if you would like to request a new feature, you can contact the maintainers of the planetscope package by emailing \sphinxhref{mailto:support@planetscope.org}{support@planetscope.org}.

\end{itemize}

\sphinxAtStartPar
We strive to provide timely and helpful support for the planetscope package and appreciate your feedback and contributions.


\chapter{FAQs}
\label{\detokenize{supportfaqs:faqs}}
\sphinxAtStartPar
This section covers some common questions and issues that users may encounter while using the planetscope package.
Error while loading data

\sphinxAtStartPar
Q: I am getting an error when trying to load a PlanetScope image file using the read function. What could be causing this?

\sphinxAtStartPar
There are a few possible reasons why you may encounter an error when trying to load a PlanetScope image file:
\begin{itemize}
\item {} 
\sphinxAtStartPar
The file path is incorrect or the file does not exist. Make sure that you are providing the correct path to the image file and that the file exists on your system.

\item {} 
\sphinxAtStartPar
The file is not in a supported format. The planetscope package can only read TIFF files. If you are trying to load a file in a different format, you will need to convert it to TIFF first.

\item {} 
\sphinxAtStartPar
The file is damaged or corrupt. If the file has been damaged or is otherwise not readable, you may get an error when trying to load it.

\end{itemize}

\sphinxAtStartPar
To troubleshoot the error, you can try the following:
\begin{itemize}
\item {} 
\sphinxAtStartPar
Check the file path and make sure that it is correct and that the file exists on your system.

\item {} 
\sphinxAtStartPar
Check the file format and make sure that it is a TIFF file.

\item {} 
\sphinxAtStartPar
Try opening the file with another tool, such as a TIFF viewer or editor, to see if it is readable.

\end{itemize}

\sphinxAtStartPar
If you are still unable to load the file after trying these steps, you may need to try a different file or contact support for further assistance.

\sphinxAtStartPar
Q: How do I generate training data from a PlanetScope image using the planetscope package?

\sphinxAtStartPar
To generate training data from a PlanetScope image using the planetscope package, you can use the following steps:
\begin{quote}

\sphinxAtStartPar
Load the image data using the read function.
Extract the data for the feature(s) you want to use for training. For example, you might use the extract\_land\_cover function to extract land cover types from the image.
Format the data as training data using the format\_training\_data function. This function expects a 2D array of data and will return X and y arrays suitable for use with a machine learning model.
\end{quote}

\sphinxAtStartPar
Here is an example of how to generate training data for land cover classification using the planetscope package:

\begin{sphinxVerbatim}[commandchars=\\\{\}]
\PYG{k+kn}{import} \PYG{n+nn}{planetscope} \PYG{k}{as} \PYG{n+nn}{ps}
\PYG{c+c1}{\PYGZsh{}Read in the image data}
\PYG{n}{data}\PYG{p}{,} \PYG{n}{metadata} \PYG{o}{=} \PYG{n}{ps}\PYG{o}{.}\PYG{n}{read}\PYG{p}{(}\PYG{l+s+s2}{\PYGZdq{}}\PYG{l+s+s2}{path/to/image.tif}\PYG{l+s+s2}{\PYGZdq{}}\PYG{p}{)}
\PYG{c+c1}{\PYGZsh{}Extract land cover types}
\PYG{n}{land\PYGZus{}cover} \PYG{o}{=} \PYG{n}{ps}\PYG{o}{.}\PYG{n}{extract\PYGZus{}land\PYGZus{}cover}\PYG{p}{(}\PYG{n}{data}\PYG{p}{)}
\PYG{c+c1}{\PYGZsh{}Format as training data}
\PYG{n}{X}\PYG{p}{,} \PYG{n}{y} \PYG{o}{=} \PYG{n}{ps}\PYG{o}{.}\PYG{n}{format\PYGZus{}training\PYGZus{}data}\PYG{p}{(}\PYG{n}{land\PYGZus{}cover}\PYG{p}{)}
\end{sphinxVerbatim}

\sphinxAtStartPar
For more information on generating training data with the planetscope package, see the documentation for the :mod:planetscope.training\_data module.


\chapter{Welcome to PlanetScope documentation}
\label{\detokenize{index:welcome-to-planetscope-documentation}}
\begin{sphinxadmonition}{warning}{Warning:}
\sphinxAtStartPar
PlanetScope is under development. The syntax of classes and methods in the library is subject to change in future releases, which will also significantly optimize performance and speed of some functionalities.
\end{sphinxadmonition}

\noindent{\sphinxincludegraphics[width=400\sphinxpxdimen]{{planetscope}.png}\hspace*{\fill}}

\sphinxAtStartPar
The planetscope package is a tool specifically designed to read and plot data from the PlanetScope satellite imaging system. The PlanetScope satellite imaging system is a collection of small, remote sensing satellites that capture high\sphinxhyphen{}resolution images of the Earth’s surface. These images are used for a variety of applications, including mapping, land use analysis, and disaster response.

\sphinxAtStartPar
The planetscope package allows users to easily access and visualize this data by providing functions for reading and parsing the raw data files, as well as functions for generating plots and maps from the data. The package also includes tools for processing the data, such as image cropping, resampling, and band math.

\sphinxAtStartPar
In addition to its visualization capabilities, the planetscope package also includes functions for generating training data for machine learning algorithms. These functions allow users to extract specific features from the satellite images, such as land cover types or building footprints, and format them in a way that is suitable for use as training data. This can be particularly useful for tasks such as land use classification or object detection.

\sphinxAtStartPar
PlanetScope is a satellite imagery provider that offers high\sphinxhyphen{}resolution imagery of the Earth’s surface for various industries, including agriculture, construction, and environmental monitoring. A Python package that has been developed to read and plot PlanetScope data might include the following features:
\begin{itemize}
\item {} 
\sphinxAtStartPar
Reading in and parsing PlanetScope data: This package would likely include functions for reading in and parsing PlanetScope data from various file formats, such as GeoTIFF.

\item {} 
\sphinxAtStartPar
Visualizing PlanetScope data: The package might include functions for visualizing the satellite imagery, such as plotting the images on a map or creating an animation of a time series of images.

\item {} 
\sphinxAtStartPar
Extracting features from PlanetScope data: The package might include functions for extracting features from the satellite imagery, such as calculating the mean or standard deviation of pixel values within a region of interest.

\item {} 
\sphinxAtStartPar
Generating training data for machine learning algorithms: The package might include functions for generating training data for machine learning algorithms by extracting features from the satellite imagery and labeling them according to a specific task, such as land cover classification.

\end{itemize}

\sphinxAtStartPar
Overall, a Python package for reading and plotting PlanetScope data could be useful for a wide range of applications, including agriculture, construction, and environmental monitoring. It could also be used to train machine learning algorithms for tasks such as land cover classification.


\chapter{Indices and tables}
\label{\detokenize{index:indices-and-tables}}\begin{itemize}
\item {} 
\sphinxAtStartPar
\DUrole{xref,std,std-ref}{genindex}

\item {} 
\sphinxAtStartPar
\DUrole{xref,std,std-ref}{modindex}

\item {} 
\sphinxAtStartPar
\DUrole{xref,std,std-ref}{search}

\end{itemize}


\renewcommand{\indexname}{Python Module Index}
\begin{sphinxtheindex}
\let\bigletter\sphinxstyleindexlettergroup
\bigletter{p}
\item\relax\sphinxstyleindexentry{planetscope}\sphinxstyleindexpageref{_autosummary/planetscope:\detokenize{module-planetscope}}
\item\relax\sphinxstyleindexentry{planetscope.ahps}\sphinxstyleindexpageref{_autosummary/planetscope.ahps:\detokenize{module-planetscope.ahps}}
\item\relax\sphinxstyleindexentry{planetscope.coast}\sphinxstyleindexpageref{_autosummary/planetscope.coast:\detokenize{module-planetscope.coast}}
\item\relax\sphinxstyleindexentry{planetscope.mymodule1}\sphinxstyleindexpageref{_autosummary/planetscope.mymodule1:\detokenize{module-planetscope.mymodule1}}
\item\relax\sphinxstyleindexentry{planetscope.mymodule2}\sphinxstyleindexpageref{_autosummary/planetscope.mymodule2:\detokenize{module-planetscope.mymodule2}}
\item\relax\sphinxstyleindexentry{planetscope.mymodule3a}\sphinxstyleindexpageref{_autosummary/planetscope.mymodule3a:\detokenize{module-planetscope.mymodule3a}}
\item\relax\sphinxstyleindexentry{planetscope.mysubpackage}\sphinxstyleindexpageref{_autosummary/planetscope.mysubpackage:\detokenize{module-planetscope.mysubpackage}}
\item\relax\sphinxstyleindexentry{planetscope.mysubpackage.mymodule3}\sphinxstyleindexpageref{_autosummary/planetscope.mysubpackage.mymodule3:\detokenize{module-planetscope.mysubpackage.mymodule3}}
\item\relax\sphinxstyleindexentry{planetscope.mysubpackage.mysubsubpackage}\sphinxstyleindexpageref{_autosummary/planetscope.mysubpackage.mysubsubpackage:\detokenize{module-planetscope.mysubpackage.mysubsubpackage}}
\item\relax\sphinxstyleindexentry{planetscope.mysubpackage.mysubsubpackage.mymodule4}\sphinxstyleindexpageref{_autosummary/planetscope.mysubpackage.mysubsubpackage.mymodule4:\detokenize{module-planetscope.mysubpackage.mysubsubpackage.mymodule4}}
\end{sphinxtheindex}

\renewcommand{\indexname}{Index}
\printindex
\end{document}